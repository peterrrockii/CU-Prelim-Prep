\newpage
\setcounter{section}{8}
\section{The Fundamental Group}
\vs

\textbf{Note:} We will consider all maps to be continuous from this point forward unless otherwise stated.

%==========================================================================
%                    SECTION 51
%==========================================================================      
\subsection{Homotopy of Paths}\nl
\setcounter{section}{51}
\setcounter{thm}{0}


\dfn If $f$ and $f^\p$ are maps of the space $X$ into the space $Y$, we say that $f$ is \textbf{homotopic} to $f^\p$ if there is a continuous map $F:X\times I\ra Y$ ($I = [0,1]$)such that
\[F(x, 0) = f(x) \qquad F(x,1) = f^\p(x)\]
for each $x$. The map $F$ is called a \textbf{homotopy} between $f$ and $f^\p$. If $f$ is homotopic to $f^\p$, we write $f\simeq f^\p$. If $f\simeq f^\p$, and $f^\p$ is a constant map, we say that $f$ is \textbf{nullhomotopic}.

\vs

\dfn Two paths $f$ and $f^\p$, mapping the interval $I = [0,1]$ into $X$, are said to be \textbf{path homotopic} if they have the same initial point $x_0$ and the same final point $x_1$, and if there is a continuous map $F:I\times I\ra X$ such that
\begin{align*}
    F(s, 0) = f(s) \qquad&\text{and}\qquad F(s,1) = f^\p(s),\\
    F(0,t) = x_0 \qquad&\text{and}\qquad F(1,t) = x_1,
\end{align*}
for each $x\in I$ and for each $t\in I$. We call $F$ a \textbf{path homotopy} between $f$ and $f^\p$. If $f$ is path homotopic to $f^\p$, we write $f\simeq_p f^\p$.

\vs

\begin{lem}
The relations $\simeq$ and $\simeq_p$ are equivalence relations
\end{lem}

\vs

\textbf{Example:} Take $n \geq 1$ and consider $[S^n, S^0]$ = set of equivalence classes of $Maps(S^n, S^0)/h.e.$ where $h.e.$ is the homotopic equivalence relation.

We have that $S^0 = \{-1,1\}$. This gives us that $[S^n, S^0] = \{*_{-1}, *_{1}\}$ where $*_{-1} = [f_{-1}:S^n \ra -1\in S^0]$ and $*_{1} = [f_{1}:S^n\ra 1\in S^1]$.

\vs

\dfn If $f$ is a path in $X$ from $x_0$ to $x_1$, and if $g$ is a path in $X$ from $x_1$ to $x_2$, we define the \textbf{product} $f*g$ of $f$ and $g$ to be the path $h$ given by the equations
\[h(x) = \begin{cases} f(2s) & \text{for } s\in [0,\frac{1}{2}],\\ g(2s - 1) & \text{for }s\in [\frac{1}{2}, 1].\end{cases}\]

\vs

\begin{thm}
The operation $*$ on paths in $X$ behaves as a group operation.
\end{thm}





%==========================================================================
%                    SECTION 52
%==========================================================================      
\subsection{The Fundamental Group}\nl
\setcounter{section}{52}
\setcounter{thm}{0}


\dfn Let $(X, \TT)$ and $(Y, \WW)$ be topological, let $A\subset X$ be closed, and let $f,g:X\ra Y$ be maps that agree on $A$. Then we say $f{\lower 4.5pt \hbox{$\widetilde{\text{\tiny rel. A}}$}}\ g$ (\textbf{homotopic relative to $\boldsymbol A$}) if
\[H|_{X\times \{0\}} = f\quad H|_{X\times \{1\}} = g\quad \text{and}\quad H_{A\times\{t\}} = f|_A = g|_A\]
where $H$ is a homotopy.

\textbf{Note:} The set $A$ can, an often will, just be a point.

\dfn A \textbf{loop} is a map $f:O\ra Y$ such that $f(0) = f(1)$.

\vs

\dfn Let $\ff = \{\text{all loops centered at }x_0\}$. Then define, ${\lower 4.5pt \hbox{$\widetilde{\text{\tiny rel. \{0,1\}}}$}}$ as an equivalence relation on $\ff$. We say that the \textbf{fundamental group} of $X$ relative to the point $x_0$ is $\pi_1(X, x_0) = X/{\lower 4.5pt \hbox{$\widetilde{\text{\tiny rel. \{0,1\}}}$}}$.

\vs

\dfn Let $\al$ be a path in $X$ from $x_0$ to $x_1$. We define a map
\[\hat \al :\pi_1(X, x_0)\ra \pi_1(X, x_1)\]
by the equation
\[\hat \al([f]) = [\al\inv]*[f]*[\al].\]

\vs

\begin{thm}
The map $\hat \al$ is a group isomorphism.
\end{thm}

\vs

\textbf{Theorem (Class):} Let $(X, \TT)\ (Y, \WW)$ be topological spaces. Then fix $x_0\in X,\ y_0\in Y$ and consider $\pi_1(X, x_0)$, and $\pi_1(Y, y_0)$ (Recall that the group operation on these groups is concatenation of loops denoted $\al * \be$  standing for $[\al]*[\be] = [\al*\be]$). If $f:X\ra Y:x_0 \mapsto y_0$, then $f$ induces a map $f_*:\pi_1(X,x_0) \ra \pi_1(Y,y_0)$ which is a homeomorphism.

\vs 

\textbf{Theorem (Class)} If $(X, \TT)$ is path connected, then for all $x_0, x_1\in X$, $\pi_1(X, x_0)\cong \pi_1(X, x_1)$.

\vs

\dfn A space $X$ is said to be \textbf{simply connected} if it is a path-connected space and if $\pi_1(X, x_0)$ is the trivial group for some $x_0\in X$, and hence for every $x_0\in X$. We often express the fact that $\pi_1(X, x_0)$ is the trivial group by writing $\pi_1(X, x_0) = 0$.

\vs

\begin{lem}
In a simply connected space $X$, any two paths having the same initial and final points are homotopic.
\end{lem}

\vs

\dfn Let $h:(X, x_0) \ra (Y, y_0)$ be a continuous map. Define
\[h_*:\pi_1(X, x_0)\ra \pi_1(Y, y_0)\]
by the equation
\[h_*([f]) = [h\circ f].\]
The map $h_*$ is called the \textbf{homomorphism induced by $\boldsymbol{h}$}, relative to the base point $x_0$.

\vs

\begin{thm}
If $h:(X, x_0)\ra (Y, y_0)$ and $k:(Y, y_0)\ra (Z, z_0)$ are continuous, then $(k\circ h)_* = k_*\circ h_*$. If $id:(X, x_0)\ra (X, x_0)$ is the identity map, then $id_*$ is the identity homomorphism.
\end{thm}

\vs

\begin{cor}
If $h:(X, x_0)\ra (Y, y_0)$ is a homeomorphism of $X$ with $Y$, then $h_*$ is an isomorphism of $\pi_1(X, x_0$ with $\pi_1(Y, y_0)$.
\end{cor}



%==========================================================================
%                    SECTION 53
%==========================================================================      
\subsection{Covering Spaces}\nl
\setcounter{section}{53}
\setcounter{thm}{0}

\textbf{Standing Hypotheses:} All topological spaces are Hausdorff, path connected, and locally path connected.

\vs

\dfn Let $p: E\ra B$ be a continuous, surjective map. The open set $U$ of $B$ is said to be \textbf{evenly covered} by $p$ if the inverse image $p\inv(U)$ can be written as the union of disjoint open sets $V_\al$ in $E$ such that for each $\al$, the restriction of $p$ to $V_\al$ is a homeomorphism of $V_\al$ onto $U$. The collection $\{V_\al\}$ will be called a partition of $p\inv(U)$ into \textbf{slices}.

\vs

\dfn \textbf{(Farsi)} Let $X$ and $\td X$ be topological spaces, and consider the map $\pi:\td X\ra  X$. We say that $\pi$ is a cover if:
\begin{enumerate}
    \item $\pi$ is surjective
    \item These is an open cover of $X$, $\UU = \{U_\al\}_{\al\in \Lambda}$ such that 
    \[\pi\inv(U_\al) = \bigsqcup_{z\in Z}\td U_{\al, z}\]
    where the $\td U_{\al, z}$ are open in $\td X$ and $\pi|_{\td U_{\al, z}}\ra U_\al$ is a homeomorphism.
    \item[2.$^\p$] For all $x\in X$, there exists $U_x$ open such that 
    \[\pi\inv(U_x) = \bigsqcup_{z\in Z} \td U_{x, z}\]
    which restricts to a homeomorphism.
\end{enumerate}

\vs

\textbf{Note:} $\pi\inv(x)$ is discrete since
\[\pi\inv(U_x) = \bigsqcup_{z\in Z} \td U_{x, z}\]

\vs

\textbf{Class Example:} Take $\psi:S^1\ra S^1: z\mapsto z^3$. This is a 3-fold cover.

\vs

\dfn Let $p:E\ra B$ be continuous and surjective. If every point $b\in B$ has a neighborhood $U$ that is evenly covered by $p$, then $p$ is called a \textbf{covering map}, and $E$ is said to be a \textbf{covering space} of $B$.

\vs

\textbf{Example:} Consider $\pi:\R \ra S^1:t \mapsto e^{2\pi i t}$. Fix $x = 1$ and take $U_1 = S^1 \backslash \{-1\}$. Then we have that 
\[\pi\inv(U_k) = \lp -\frac{1}{2}, \frac{1}{2}\rp + k\qquad k \in \Z.\]

This gives us that $\pi|_{\td U_k}:\td U_k\ra U_k$ is a homeomorphism by calculus using $\ln$.







%==========================================================================
%                    SECTION 54
%==========================================================================      
\subsection{The Fundamental Group of the Circle}\nl
\setcounter{section}{54}
\setcounter{thm}{0}


\dfn Let $p:E \ra B$ be a map. If $f$ is a continuous mapping of some space $X$ into $B$, a \textbf{lifting} of $f$ is a map $\tilde f:X\ra E$ such that $p\circ \tilde f = f$.
\begin{center}
    \begin{tikzcd}
     & E\arrow[d, "p"]\\
     X \arrow[ur, "\tilde f"] \arrow[r, "f", swap] & B
    \end{tikzcd}
\end{center}

\vs

\begin{lem}\textbf{(Path Lifting Lemma)}
Let $p:E\ra B$ be a covering map, let $p(e_0) = b_0$. Any path $f:[0,1] \ra B$ beginning at $b_0$ has a unique lifting to a path $\tilde f$ in $E$ beginning at $e_0$.
\end{lem}

\vs

\textbf{Path lifting (Farci)} Let $\pi: \td X \ra X$ be a covering map. Then let $\al: I \ra X$ be a path in $X$, and fix $\tilde e\in \td X$, $\tilde e \in \pi\inv(\al(0))$. Then there exists a unique lift $\tilde\al$, $\tilde \al:I \ra \td X$ such that $\tilde \al (0) = \tilde e$.

\vs

\textbf{Corollary (Farsi)} Given $pi:\tilde X \ra X$ a covering map, then the Path Lifting Lemma gives a function from a loop $\al$ to $\pi\inv(x_0)$ where $x_0 = \al(0)$ where $\al \mapsto \tilde \al(1)$.

\vs

\begin{lem}
Let $p: E \ra B$ be a covering map; let $p(e_0) = b_0$. Let the map $F: I \times I \ra B$ be continuous, with $F(0,0) = b_0$. There is a unique lifting $F$ to a continuous map
\[\td F: I\times I \ra E\]
such that $\td F(0,0) = e_0$. If $F$ is a path homotopy, then $\tilde F$ is a path homotopy.
\end{lem}

\vs

\textbf{Homotopy Lifting (Farci)} Let $p: \td X \ra X:\tilde x_0 \mapsto x_0$ be a covering map. Define a map $F: I \times I \ra X$ with $x_0 = \al(0) = \be(0)$ and 
\[F(x, 0) = \al(s) \qquad\text{and}\qquad F(s, 1) = \be(s).\]
Then there exists a unique homotopy of paths $\td F: I\times I \ra \td X$ such that
\[\td F(0,0) = \tilde\al(0) = \tilde x_0.\]



\vs 


\textbf{Theorem (Farci)} $\pi_1(S^1, 1) \cong \Z$.

\begin{proof}
Let $S: \{\text{Loops in $S^1$ at 1} \}\ra \Z:\al \mapsto \tilde \al(1)$. We can then define $\wh S: \pi_1 (S^1, 1) \ra \Z$ using the homotopy lifting lemma......
\end{proof}













%eof