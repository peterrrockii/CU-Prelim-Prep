%################################################################################
\section{Field Theory}
\setcounter{thm}{0}

\begin{defn}
The \textit{characteristic} of a field $F$, denoted $ch(F)$, is defined to be the smallest positive integer $p$ such that $p\cdots 1_F = 0$ if such a $p$ is defined to be 0 otherwise. 
\end{defn}

\nl

\begin{prop}
The characteristic of a field $F$, $ch(F)$ is either 0 or a prime $p$. If $ch(F) = p$ then for any $\al \in F$,
\[p\cdot \al = \underbrace{\al + \al + \cdots + \al}_{\text{$p$ times}} = 0.\]
\end{prop}

\nl

\begin{defn}
The \textit{\textbf{prime subfield}} of a field $F$ is the subfield of $F$ generated by the multiplicative identity $1_F$ of $F$. It is (isomorphic to) either $\Q$ or $\F_p$.
\end{defn}

\nl

\begin{defn}
If $K$ is a field containing the subfield $F$, then $K$ is said to be an \textit{\textbf{extension field}} of $F$, denoted $K/F$ or by the digram 
\begin{center}
\begin{tikzcd}[column sep = 1em, row sep = 1.5em]
K \arrow[dash, d]\\
F
\end{tikzcd}
\end{center}
In particular, every field $F$ is an extension of its prime subfield. The field $F$ is sometimes called the \textit{\textbf{base field}} of the extension.
\end{defn}

\nl

\begin{defn}
The \textit{\textbf{degree}} (or \textit{\textbf{relative degree}} or \textit{\textbf{index}}) of a field extension $K/F$, denoted $[K:F]$, is the dimension of $K$ as a vector space over $F$. The extension is said to be \textit{\textbf{finite}} if the degree of $K$ is finite and infinite otherwise.
\end{defn}

\nl

\begin{prop}
Let $\vphi: F\ra F^\p$ be a homomorphism of fields. Then $\vphi$ is either identically 0 or is injective, so that the image of $\vphi$ is either 0 or isomorphic to $F$.
\end{prop}

\nl

\begin{thm}
Let $F$ be a field and let $p(x)\in F[x]$ be an irreducible polynomial. Then there exists a field $K$ containing an isomorphic copy of $F$ in which $p(x)$ has a root. Identifying $F$ with this isomorphic copy show that there exists an extension of $F$ in which $p(x)$ has a root.
\end{thm}

\nl

\begin{thm}
Let $p(x)\in F[x]$ be an irreducible polynomial of degree $n$ over the field $F$ and let $K$ be the field $F[x]/(p(x))$. Let $\theta = x\mod(p(x))\in K$. Then the elements 
\[1, \theta, \theta^2,\ldots, \theta^{n-1}\]
are a basis for $K$ as a vector space over $F$, so the degree of the extension is $n$, i.e., $[K:F] = n$. Hence
\[K = \{a_0 + a_1\theta + a_2\theta^2 + \cdots + a_{n - 1}\theta^{n - 1}\ |\ a_0, a_1, \ldots, a_{n - 1}\in F\}\]
consists of all polynomials of degree $<n$ in $\theta$.
\end{thm}

\nl

\begin{cor}
Let $K$ be as in the previous theorem, and let $a(\theta), b(\theta)\in K$. be two polynomials of degree $<n$ in $\theta$. Then addition in $K$ is defined simply by the usual polynomial addition and multiplication in $K$ is defined by
\[a(\theta)b(\theta) = r(\theta)\]
where $r(x)$ is the remainder obtained after dividing the polynomial $a(x)b(x)$ by $p(x)$ in $F[x]$.
\end{cor}

\nl

\begin{defn}
Let $K$ be an extension of the field $F$ and let $\al,\be,\ldots\in K$ be a collection of elements of $K$. Then the smallest subfield of $K$ containing both $F$ and the elements of $\al,\be,\ldots$ denoted $F(\al,\be,\ldots)$ is called the field \textbf{\textit{generated by $\al,\be,\ldots$ over $F$}}.
\end{defn}

\nl

\begin{defn}
If the field $K$ is generated by a single element $\al$ over $F$, $K = F(\al)$, then $K$ is said to be a \textit{\textbf{simple}} extension of $F$ and the element $\al$ is called a \textit{\textbf{primitive element}} for the extension.
\end{defn}

\nl

\begin{thm}
Let $F$ be a field and let $p(x) \in F[x]$ be an irreducible polynomial. Suppose $K$ is an extension field of $F$ containing a root $\al$ of $p(x)$: $p(\al) = 0$. Let $F(\al)$ denote the subfield of $K$ generated over $F$ by $\al$. Then
\[F(\al) \cong F[x]/(p(x)).\]
\end{thm}

\nl

\begin{cor}
Suppose in the previous theorem that $p(x)$ is of degree $n$. Then
\[F(\al) = \{a_0 + a_1\al + a_2\al^2 + \cdots + a_{n - 1}\al^{n - 1} \ |\ a_0, a_1, \ldots a_{n - 1}\}\seq K.\]
\end{cor}

\nl

\begin{thm}
Let $\vphi: F\overset{\sim}{\ra} F^\p$ be an isomorphism of fields. Let $p(x)\in F[x]$ be an irreducible polynomial and let $p^\p(x)\in F^\p[x]$ be the irreducible polynomial obtained by applying the map $\vphi$ to the coefficients of $p(x)$. Let $\al$ be a root of $p(x)$ and let $\be$ be a root of $p^\p(x)$. Then there is an isomorphism 
\[\sigma:F(\al)\overset{\sim}\longrightarrow F^\p(\be)\]
\[\al\longmapsto \be\]
\end{thm}

\nl

\begin{defn}
The element $\al\in K$ is said to be \textit{\textbf{algebraic}} over $F$ if $\al$ is a root of some nonzero polynomial $f(x)\in F[x]$. If $\al$ is not algebraic over $F$ then $\al$ is said to be \textit{\textbf{transcendental}} over $F$. The extension $K/F$ is said to be \textit{algebraic} if every element of $K$ is algebraic over $F$.
\end{defn}

\nl

\hl{\textbf{Note:}} If $K$ is algebraic then it is not necessarily true that $K$ is finite. Consider the set $A$ of all algebraic numbers over $\Q$. Then $\Q(A)$ is algebraic but is certainly not finite.

\nl

\begin{prop}
Let $\al$ be algebraic over $F$. Then there is a unique, monic, irreducible polynomial $m_{\al,F}(x)\in F[x]$ which has $\al$ as a root. A polynomial $f(x)\in F[x]$ has $\al$ as a root if and only if $m_{\al, F}(x)$ divides $f(x)$ in $F[x]$. 
\end{prop}

\nl

\begin{cor}
If $L/F$ is an extension of fields and $\al$ is algebraic over both $F$ and $L$, then $m_{\al,L}(x)$ divides $m_{\al,F}(x)$ in $L[x]$.
\end{cor}

\nl

\begin{defn}
The polynomial $m_{\al, F}(x)$ is called the \textit{\textbf{minimal polynomial}} for $\al$ over $F$. The \textit{degree} of $m_\al(x)$ is called the \textit{degree} of $\al$.
\end{defn}

\nl

\begin{prop}
Let $\al$ be algebraic over the field $F$ and let $F(\al)$ be the field generated by $\al$ over $F$. Then
\[F(\al) \cong F[x]/(m_\al(x))\]
so that in particular
\[[F(\al):F] = \deg(m_\al(x)) = \deg(\al),\]
i.e., the degree of $\al$ over $F$ is the degree of the extension it generates over $F$.
\end{prop}

\nl

\begin{prop}
The element $\al$ is algebraic over $F$ if and only if the simple extension $F(\al)/F$ is finite.
\end{prop}

\nl

\begin{cor}
If the extension $K/F$ is finite, then it is algebraic.
\end{cor}

\nl

\begin{thm}
Let $F\seq K\seq L$ be fields. Then 
\[[L:F] = [L:K][K:F].\]
\end{thm}

\nl

\begin{cor}
Suppose $L/F$ is a finite extension and let $K$ be any subfield of $L$ containing $F$, $F\seq K\seq L$. Then $[K:F]$ divides $[L:F]$.
\end{cor}

\nl

\begin{defn}
An extension $K/F$ is \textbf{\textit{finitely generated}} if there are elements $\al_1,\al_2,\ldots,\al_k$ in $K$ such that $K = F(\al_1,\al_2,\ldots,\al_k)$.
\end{defn}

\nl

\begin{lem}
$F(\al,\be)= (F(\al))(\be)$.
\end{lem}

\nl

\begin{thm}
The extension $K/F$ is finite if an only if $K$ is generated by a finite number of algebraic elements over $F$. More precisely, a field generated over $F$ by a finite number of algebraic elements of degrees $n_1,n_2,\ldots, n_k$ is algebraic of degree $\leq n_1n_2\cdots n_k$.
\end{thm}

\nl

\begin{cor}
Suppose $\al$ and $\be$ are algebraic over $F$. Then $\al\pm\be,\ \al\be,\ \al/\be$ (for $\be\neq 0$), are all algebraic.
\end{cor}

\nl

\begin{cor}
Let $L/F$ be an arbitrary extension. Then the collection of elements of $L$ that are algebraic over $F$ form a subfield $K$ of $L$.
\end{cor}

\nl

\begin{thm}
If $K$ is algebraic over $F$ and $L$ is algebraic over $K$, then $L$ is algebraic over $F$.
\end{thm}

\nl

\begin{defn}
Let $K_1$ and $K_2$ be two subfields of a field $K$. Then the \hl{\textit{\textbf{composite field}}} of $K_1$ and $K_2$, denoted $K_1K_2$, is the smallest subfield of $K$ containing both $K_1$ and $K_2$. Similarly, the composite of any collection of subfields of $K$ is the smallest subfield containing all the subfields.
\end{defn}

\nl

\begin{prop}
Let $K_1$ and $K_2$ be two finite extensions of a field $F$ contained in $K$. Then
\[[K_1K_2:F]\leq[K_1:F][K_2:F]\]
with equality if and only if an $F$-basis for one of the fields remains linearly independent over the other field. If $\al_1,\al_2,\ldots,\al_n$ and $\be_1,\be_2,\ldots,\be_m$ are bases for $K_1$ and $K_2$ over $F$, respectively, then the elements $\al_i\be_j$ for $i = 1..n$ and $j = 1..m$ span $K_1K_2$ over $F$.
\end{prop}

By this proposition, we have the following diagram

\begin{center}
\begin{tikzcd}[column sep = 2em, row sep = 2em]
 & K_1K_2\arrow[dash, dr, "\leq n"]\arrow[dash, dl, "\leq m", swap] & \\
K_1\arrow[dash, dr, "n", swap] & & K_2\arrow[dash, dl, "m"]\\
 & F &
\end{tikzcd}
\end{center}

\nl

\begin{cor}
Suppose that $[K_1:F] = n,\ [K_2: F] = m$ in the previous proposition, where $m$ and $n$ are relatively prime. Then $[K_1K_2 : F] = [K_1:F][K_2:F]$.
\end{cor}

\nl

\begin{prop}
If the element $\al\in \R$ is obtained from a field $F\subset \R$ by a series of straightedge and compass constructions then $[F(\al) : F] = 2^k$ for some integer $k\geq 0$.
\end{prop}

\nl

\begin{defn}
The extension field $K$ of $F$ is called a \textit{\textbf{splitting field}} for the polynomial $f(x)\in F[x]$ if $f(x)$ factors completely in $K[x]$ and $f(x)$ does not factor completely into linear factors over any proper subfield of $K$ containing $F$.
\end{defn}

\nl

\begin{thm}
For any field $F$, if $f(x)\in F[x]$ then there exists an extension $K$ of $F$ which is a splitting filed for $f(x)$.
\end{thm}

\nl

\begin{defn}
If $K$ is an algebraic extension of $F$ which is the splitting field over $F$ for a collection of polynomials $f(x)\in F[x]$ then $K$ is called a \textit{\textbf{normal extension}} of $F$.
\end{defn}

\nl

\begin{prop}
A splitting field of a polynomial of degree $n$ over $F$ is of degree at most $n!$ over $F$.
\end{prop}

\nl

\begin{defn}
A generator of the cyclic group of all $n^{th}$ roots of unity is called a \textit{\textbf{primitive}} $n^{th}$ root of unity.
\end{defn}

\nl

\begin{defn}
The field $\Q(\zeta_n)$ is called the \textit{\textbf{cyclotomic field of n$^{th}$ roots of unity}}.
\end{defn}

\nl

\begin{thm}
Let $\vphi: F\overset{\sim}{\ra} F^\p$ be an isomorphism of fields. Let $f(x)\in F[x]$ be a polynomial and let $f^\p(x)\in F^\p[x]$ be the polynomial obtained by applying $\vphi$ to the coefficients of $f(x)$. Let $E$ be a splitting field for $f(x)$ over $F$ and let $E^\p$ be a splitting field for $f^\p(x)$ over $F^\p$. Then the isomorphism $\vphi$ extends to an isomorphism $\sigma: E\overset{\sim}{\ra} E^\p$, i.e., $\sigma$ restricted to $F$ is the isomorphism $\vphi$:
\begin{center}
\begin{tikzcd}
\sigma:\ \  E\arrow[r, "\sim"]\arrow[dash, start anchor = {[xshift = 2.5ex]}, end anchor = {[xshift = 2.5ex]}, d] & E^\p\arrow[dash, d]\\
\vphi:\ \  F\arrow[r, "\sim"] & F^\p
\end{tikzcd}
\end{center}
\end{thm}

\nl

\begin{cor}\textit{(Uniqueness of Splitting Fields)}
Any two splitting fields for a polynomial $f(x)\in F[x]$ over a field $F$ are isomorphic.
\end{cor}

\nl

\begin{defn}
The field $\ol F$ is called an \textit{\textbf{algebraic closure}} of $F$ if $\ol F$ is algebraic over $F$ and if every polynomial $f(x)\in F[x]$ splits completely over $\ol F$.
\end{defn}

\nl

\begin{defn}
A filed $K$ is said to be \textit{\textbf{algebraically closed}} if every polynomial with coefficients in $K$ has root in $K$.
\end{defn}

\nl

\begin{prop}
Let $\ol F$ be an algebraic closure of $F$. Then $\ol F$ is algebraically closed.
\end{prop}

\nl

\begin{prop}
For any filed $F$ there exists an algebraically closed field $K$ containing $F$.
\end{prop}

\nl

\begin{prop}
Let $K$ be an algebraically closed field and let $F$ be a subfield of $K$. Then the collection of elements $\ol F$ of $K$ that are algebraic over $F$ is an algebraic closure of $F$. An algebraic closure is unique up to isomorphism.
\end{prop}

\nl

\begin{defn}
A polynomial over $F$ is called \textit{\textbf{separable}} if has no multiple roots. A polynomial which is not separable is called \textit{\textbf{inseparable}}.
\end{defn}

\nl

\begin{defn}
The \textit{\textbf{derivative}} of the polynomial
\[f(x) = a_nx^n + a_{n - 1}x^{n - 1} + \cdots + a_1x + a_0\in F[x]\]
is defined to be the polynomial
\[D_xf(x) = na_n x^{n - 1} + (n - 1)a_{n - 1}x^{n - 2} + \cdots + 2a_2x + a_1 \in F[x].\]
\end{defn}

\nl

\begin{prop}
\hl{A polynomial $f(x)$ has a multiple root $\al$ if and only if $\al$ is also a root of $D_xf(x)$.} In particular, $f(x)$ is separable if and only if it is relatively prime to its derivative $\gcd(f(x), D_xf(x)) = 1$.
\end{prop}

\nl

\begin{cor}
Every \textit{irreducible} polynomial over a field of characteristic 0 is separable. A polynomial over such a field is separable if and only if it is the product of distinct irreducible polynomials.
\end{cor}

\nl

\begin{prop}
Let $F$ be a field of characteristic $p$. Then for any $a,b\in F$,
\[(a + b)^p = a^p + b^p,\quad\text{and}\quad(ab)^p = a^pb^p.\]
Put another way, the $p^{th}$-power map defined by $\vphi(a) = a^p$ is an injective field homomorphism from $F$ to $F$. This map is called the \textit{\textbf{Frobenius endomorphism}} of $F$.
\end{prop}

\nl

\begin{cor}
Suppose that $\F$ is a finite field of characteristic $p$. Then every element of $\F$ is a $p^{th}$ power in $\F$ (notationally $\F = \F^p$).
\end{cor}

\nl

\begin{prop}
Every irreducible polynomial over a finite field $\F$ is separable. A polynomial in $\F[x]$ is separable if an only if it is the product of distinct irreducible polynomials in $\F[x]$.
\end{prop}

\nl

\begin{defn}
A filed $K$ of characteristic $p$ is called \textit{\textbf{perfect}} if every element of $K$ is a $p^{th}$ power in $K$, i.e., $K = K^p$. Any field of characteristic 0 is also called perfect.
\end{defn}

\nl

\begin{prop}
Let $p(x)$ be an irreducible polynomial over a field $F$ of characteristic $p$. Then there is a unique integer $k\geq 0$ ans a unique irreducible, separable polynomial $p_{sep}(x)\in F[x]$ such that
\[p(x) = p_{sep}(x^{p^k}). \]
\end{prop}

\nl

\begin{defn}
Let $p(x)$ be an irreducible polynomial over field of characteristic $p$. The degree $p_{sep}(x)$ in the last proposition is called the \textit{\textbf{inseparable degree}} of $p(x)$, denoted $deg_i(p(x))$.
\end{defn}

\nl

\begin{defn}
The field $K$ is said to \textit{\textbf{separable}} over $F$ if every element of $K$ is the root of a separable polynomial over $F$. A field which is not separable is \textit{\textbf{inseparable}}.
\end{defn}

\nl

\begin{cor}
Every finite extension of a perfect field is separable. In particular, every finite extension of either $\Q$ or a finite field is separable.
\end{cor}

\nl

\begin{defn}
Let $\mu_n$ denote that \textit{\textbf{group of n$^{\textbf{\textit{th}}}$ roots of unity over $\Q$}}.
\end{defn}

\nl

\begin{defn}
Define the $n^{th}$ \textit{\textbf{cyclotomic polynomial}} $\Phi_n(x)$ to be the polynomial whose roots are primitive $n^{th}$ roots of unity:
\[\Phi_n(x) = \prod_{\zeta\text{ primitive }\in\ \mu_n}(x - \zeta) = \prod_{\substack{1 \leq a \leq n \\ (a,n) = 1}} (x - \zeta_n^a)\]
(which is of degree $\vphi(n)$ for the Euler $\vphi$).
\end{defn}

\nl

\begin{lem}
The cyclotomic polynomial $\Phi_n(x)$ is a monic polynomial in $\Z[x]$ of degree $\vphi(n)$.
\end{lem}

\nl

\begin{thm}
The cyclotomic polynomial $\Phi_n(x)$ is an irreducible, monic polynomial in $\Z[x]$ of degree $\vphi(n)$.
\end{thm}

\nl

\begin{cor}
The degree over $\Q$ of the cyclotomic field of $n^{th}$ roots of unity is $\vphi(n)$:
\[[\Q(\zeta_n): \Q] = \vphi(n).\]
\end{cor}


%################################################################################

\section{Galois Theory}
\setcounter{thm}{0}

\begin{defn}\nl
\begin{enumerate}
\item An isomorphism $\sig$ of $K$ with itself is called an \textit{\textbf{automorphism}} of $K$. The collection of automorphisms of $K$ is denoted $\Aut(K)$. If $\al\in K$ we shall write $\sig\al$ for $\sig(\al)$.
\item An automorphism $\sig\in \Aut(K)$ is said to \textit{\textbf{fix}} an element $\al\in K$ if $\sig\al = \al$. If $F$ is a subset of $K$, then an automorphism $\sig$ is said to \textit{\textbf{fix}} $F$ if it fixes all the elements of $F$. 
\end{enumerate}
\end{defn}

\nl

\begin{defn}
Let $K/F$ be an extension of fields. Let $\Aut(K/F)$ be the collections of automorphisms of $K$ which fix $F$.
\end{defn}

\nl

\begin{prop}
$\Aut(K)$ is a group under composition and $\Aut(K/F)$ is a subgroup.
\end{prop}

\nl

\begin{prop}
Let $K/F$ be a field extension and let $\al \in K$ be an algebraic over $F$. Then for any $\sig\in \Aut(K/F)$, $\sig\al$ is a root of the minimal polynomial for $\al$ over $F$, i.e., $\Aut(K/F)$ permutes the roots of irreducible polynomials. Equivalently, any polynomial with coefficients in $F$ having $\al$ as a root also has $\sig\al$ as a root.
\end{prop}

\nl

\begin{prop}
Let $H\leq \Aut(K)$ be a subgroup of the group of automorphisms of $K$. Then the collection $F$ of elements of $K$ fixed by all elements of $H$ is a subfield of $K$.
\end{prop}

\nl

\begin{defn}
If $H$ is a subgroup of the group of automorphisms of $K$, the subfield of $K$ fixed by all elements of $H$ is called the \textit{\textbf{fixed field}} of $H$.
\end{defn}

\nl

\begin{prop}
The association of groups to fields and fields to groups defined above is inclusion reversing, namely
\begin{enumerate}
\item if $F_1\seq F_2\seq K$ are two subfields of $K$ then $\Aut(K/F_2)\leq \Aut(K/F_1)$, and
\item if $H_1\leq H_2\leq \Aut(K)$ are two subgroups of automorphisms with associated fixed fields $F_1$ and $F_2$, respectively, then $F_2\seq F_1$.
\end{enumerate}
\end{prop}

\nl

\begin{prop}
Let $E$ be the splitting field over $F$ of the polynomial $f(x)\in F[x]$. Then 
\[|\Aut(E/F)|\leq [E:F]\]
with equality if $f(x)$ is separable over $F$.
\end{prop}

\nl

\begin{defn}
Let $K/F$ be a finite extension. Then $K$ is said to be \textit{\textbf{Galois}} over $F$ and $K/F$ is a \textit{\textbf{Galois extension}} if $|\Aut(K/F)| = [K:F]$. If $K/F$ is Galois the group of automorphisms $\Aut(K/F)$ is called the \textit{\textbf{Galois group}} of $K/F$, denoted $\Gal(K/F)$.
\end{defn}

\nl

\begin{cor}
If $K$ is the splitting field over $F$ of a separable polynomial $f(x)$ then $K/F$ is Galois.
\end{cor}

\nl

\begin{defn}
If $f(x)$ is a separable polynomial over $F$, then the \textit{\textbf{Galois group of $f(x)$ over $F$}} is the Galois group of the splitting field of $f(x)$ over $F$.
\end{defn}

\nl

\begin{defn}
A \textit{\textbf{character}} $\chi$ of a group $G$ with values in a field $L$ is a homomorphism from $G$ to the multiplicative group of $L$:
\[\chi: G\ra L^\times\]
i.e., $\chi(g_1g_2) = \chi(g_1)\chi(g_2)$ for all $g_1,g_2\in G$ and $\chi(g)$ is a nonzero element of $L$ for all $g\in G$.
\end{defn}

\nl

\begin{defn}
The characters $\chi_1, \chi_2, \cdots, \chi_n$ of $G$ are said to be \textit{\textbf{linearly independent}} over $L$ if they are linearly independent as functions on $G$, i.e., if there is no nontrivial relation
\[a_1\chi_1 + a_2\chi_2 + \cdots + a_n\chi_n = 0\]
as a function on $G$ (that is, $a_1\chi_1 + a_2\chi_2 + \cdots + a_n\chi_n = 0$ for all $g\in G$).
\end{defn}

\nl

\begin{thm}\textit{(Linear Independence of Characters)}
If $\chi_1, \chi_2, \cdots, \chi_n$ are distinct characters of $G$ with values in $L$ then they are linearly independent over $L$.
\end{thm}

\nl

\begin{cor}
If $\sig_1, \sig_2, \ldots, \sig_n$ are distinct embeddings (injective homomorphisms) of a field $K$ into a field $L$, then they are linearly independent as functions on $K$. In particular distinct automorphisms of a field $K$ are linearly independent as functions on $K$.
\end{cor}

\nl

\begin{thm}
Let $G = \{\sig_1 = 1, \sig_2, \ldots, \sig_n\}$ be a subgorup of the automorphisms of a field $K$ and let $F$ be the fixed field. Then
\[[K:F] = n = |G|.\]
\end{thm}

\nl

\begin{cor}
Let $K/F$ be any finite extension. Then
\[|\Aut(K/F)|\leq [K:F]\]
with equality if and only if $F$ is the fixed field of $\Aut(K/F)$. Put another way, $K/F$ is Galois if and only if $F$ is the fixed field of $\Aut(K/F)$.
\end{cor}

\begin{proof}
Let $F_1$ be the fixed field of $\Aut(K/F)$. Then since every $\sigma \in \Aut(K/F)$ fixes $F$, we have that 
\[F\seq F_1\seq K.\]
By \textcolor{red}{Theorem 14.9} we then get that $[K:F_1] = |\Aut(K/F)|$. Hence $[K:F] = |\Aut(K/F)|[F_1:F]$.
\end{proof}

\nl

\begin{cor}
Let $G$ be a finite subgroup of automorphisms of a field $K$ and let $F$ be the fixed field. Then every automorphism of $K$ fixing $F$ is contained in $G$, i.e, $\Aut(K/F) = G$, so that $K/F$ is Galois, with Galois group $G$.
\end{cor}

\nl

\begin{cor}
If $G_1\neq G_2$ are distinct finite subgroups of automorphisms of a field $K$ then their fixed fields are also distinct.
\end{cor}

\nl

\begin{thm}
\hl{The extension $K/F$ is Galois if and only if $K$ is the splitting field of some separable polynomial over $F$. Furthermore, if this is the case then every irreducible polynomial with coefficients in $F$ which has a root in $K$ is separable and has all its roots in $K$ (so in particular $K/F$ is a separable extension).}
\end{thm}

\nl

\begin{defn}
Let $K/F$ be a Galois extension. If $\al\in K$ the elements $\sig \al$ for $\sig$ in $\Gal(K/F)$ are called \textit{\textbf{conjugates}} (or \textit{\textbf{Galois conjugates}}) of $\al$ over $F$. If $E$ is a subfield of $K$ containing $F$, the field $\sig(E)$ is called the \textit{\textbf{conjugate field}} of $E$ over $F$.
\end{defn}

\nl

\textbf{Note.} We now have 4 characterizations of Galois extensions $K/F$:
\begin{enumerate}
\item splitting fields of separable polynomials over $F$
\item fields where $F$ is precisely the set of elements fixed by $\Aut(K/F)$
\item fields with $[K:F] = |\Aut(K/F)|$
\item finite, normal, separable extensions.
\end{enumerate}

\nl

\begin{thm}\underline{\hl{\textbf{\textit{(Fundamental Theorem of Galois Theory)}}}}
Let $K/F$ be a Galois extension and set $G = \Gal(K/F)$. Then there is a bijection
\[ 
\left \{
\begin{tabular}{cc}
& K\\
\text{subfields } E & |\\
\text{of } K & E\\
\text{containing } F & |\\
& F
\end{tabular}
\right \}\quad \longleftrightarrow \quad 
\left \{\begin{tabular}{cc}
& 1\\
\text{subgroups } H & |\\
\text{of } G & H\\
& |\\
& G
\end{tabular}\right \}
\]
given by the correspondences
\begin{align*}
E\qquad &\longrightarrow \quad\left \{\begin{tabular}{c}
\text{the elements of } G\\
\text{fixing } E
\end{tabular}\right \}\\
\left \{\begin{tabular}{c}
\text{the fixed field}\\
\text{of } H
\end{tabular}\right \}\quad &\longleftarrow\qquad H
\end{align*}
which are inverse to each other. Under this correspondence,
\begin{enumerate}
\item (inclusion reversing) If $E_1, E_2$ correspond to $H_1, H_2$, respectively, then $E_1\seq E_2$ if and only if $H_2\leq H_1$
\item $[K:E] = |H|$ and $[E:F] = |G:H|$, the index of $H$ in $G$:
\[\begin{tabular}{ccc}
K & & \\
| & \} & |H|\\
E & & \\
| & \} & |G\ :\ H|\\
F & & 
\end{tabular}\]
\item $K/E$ is always Galois, with Galois group $\Gal(K/E) = H$:
\[\begin{tabular}{cc}
K & \\
| & H\\
E & 
\end{tabular}\]
\item $E$ is Galois over $F$ if and only $H$ is a normal subgroup in $G$. If this is the case, then the Galois group is isomorphic to the quotient group 
\[\Gal(E/F)\cong G/H.\]
More generally, even if $H$ is not necessarily normal in $G$, the isomorphisms of $E$ which fix $F$ are in one to one correspondence with the cosets $\{\sig H\}$ of $H$ in $G$.
\item If $E_1, E_2$ correspond to $H_1, H_2$, respectively, then the intersection $E_1\cap E_2$ corresponds to the group $\langle H_1, H_2\rangle$ generated by $H_1$ and $H_2$ and the composite field $E_1E_2$ corresponds to the intersection $H_1\cap H_2$. Hence the lattice of subfields of $K$ containing $F$ and the lattice of subgroups of $F$ are "dual" (the lattice diagram for one is the lattice diagram for the other turned upside down).
\end{enumerate}
\end{thm}

\nl

\begin{prop}
Any finite field is isomorphic to $\F_{p^n}$ for some prime $p$ and some integer $n\geq 1$. The field $\F_{p^n}$ is the splitting field over $\F_p$ of the polynomial $x^{p^n} - x$, with cyclic Galois group of order $n$ generated by the Frobenius automorphism $\sig_p$. The subfields of $\F_{p^n}$ are all Galois over $\F_p$ and are in one to one correspondence with the divisors $d$ of $n$. They are the fields $\F_{p^d}$, the fixed fields of $\sig_p^d$.
\end{prop}

\nl

\begin{cor}
The irreducible polynomial $x^4 + 1\in \Z[x]$ is reducible modulo every prime p.
\end{cor}

\nl

\begin{prop}
The finite field $\F_{p^n}$ is a simple extension of $\F_p$. In particular, there exists an irreducible polynomial of degree $n$ over $\F_p$ for every $n\geq 1$.
\end{prop}

\nl

\begin{prop}
The polynomial $x^{p^n} - x$ is precisely the product of all the distinct irreducible polynomials in $\F_p[x]$ of degree $d$ where $d$ runs through all the divisors of $n$.
\end{prop}

\nl

\begin{prop}
Suppose $K/F$ is a Galois extension and $F^\p/ F$ is any extension. Then $KF^\p/F^\p$ is a Galois extension, with Galois group
\[\Gal(KF^\p/F^\p)\cong \Gal(K/K\cap F^\p)\]
isomorphic to a subgroup of $\Gal(K/F)$. Pictorially,
\[
\begin{tikzcd}
 & KF^\p & \\
K\arrow[ur, dash] & & F^\p\arrow[ul, dash, "//"{anchor = center, sloped}]\\
 & K\cap F^\p\arrow[ur, dash]\arrow[ul, dash, "//"{anchor = center, sloped}] & \\
 & F\arrow[u, dash] &
\end{tikzcd}
\]
\end{prop}

\nl

\begin{cor}
Suppose $K/F$ is a Galois extension and $F^\p/F$ is any finite extension. Then 
\[[KF^\p:F] = \frac{[K:F][F^\p:F]}{[K\cap F^\p:F]}.\]
\end{cor}

\nl

\begin{prop}
Let $K_1$ and $K_2$ be Galois extensions of a field $F$. Then
\begin{enumerate}
\item The intersection $K_1\cap K_2$ is Galois over $F$.
\item The composite $K_1K_2$ is Galois over $F$. The Galois group is isomorphic to the subgroup
\[H = \{\langle \sig, \tau\rangle\ |\ \sig |_{K_1\cap K_2} = \tau|_{K_1\cap K_2}\}\]
of the direct product $\Gal(K_1/ F)\times \Gal(K_2 / F)$ consisting of elements whose restrictions to the intersection $K_1\cap K_2$ are equal.
\end{enumerate}
\[
\begin{tikzcd}
 & K_1K_2 & \\
K_1\arrow[ur, dash] & & K_2\arrow[ul, dash]\\
 & K_1\cap K_2\arrow[ur, dash]\arrow[ul, dash] & \\
 & F\arrow[u, dash] &
\end{tikzcd}
\]
\end{prop}

\nl

\begin{cor}
Let $K_1$ and $K_2$ be Galois extensions of a field $F$ with $K_1\cap K_2 = F$. Then
\[\Gal(K_1K_2/ F)\cong \Gal(K_1/ F)\times \Gal(K_2 / F).\]
Conversely, if $K$ is Galois over $F$ and $G = \Gal(K/F) = G_1\times G_2$ is the direct product of two subgroups $G_1$ and $G_2$, then $K$ is the composite of two Galois extensions $K_1$ and $K_2$ of $F$ with $K_1\cap K_2 = F$.
\end{cor}

\nl

\begin{cor}
Let $E/F$ be any finite, separable extension. Then $E$ is contained in an extension $K$ which is Galois over $F$ and is minimal in the sense that in a fixed algebraic closure of $K$ any other Galois extension of $F$ containing $E$ contains $K$.
\end{cor}

\nl

\begin{defn}
The Galois extension $K$ of $F$ containing $E$ in the previous corollary is called the \textit{\textbf{Galois closure}} of $E$ over $F$.
\end{defn}

\nl

\begin{prop}
Let $K/F$ be a finite extension. Then $K = F(\tht)$ if and only if there exist finitely many subfields of $K$ containing $F$.
\end{prop}

\nl

\begin{thm}\hl{\textit{(The Primitive Element Theorem)}} If $K/F$ is finite and separable, then $K/F$ is simple. In particular, any finite extension of fields of characteristic 0 is simple.
\end{thm}

\nl

\begin{thm}
The Galois group of the cyclotomic field $\Q(\zeta_n)$ of $n^{th}$ roots of unity s isomorphic to the multiplicative group $(\Z/n\Z)^\times$. The isomorphism is give explicitly by the map
\begin{align*}
(\Z/n\Z)^\times\  &\overset{\sim}{\longrightarrow}\ \Gal(\Q(\zeta_n)/\Q)\\
a\mod n\  &\longmapsto\  \sig_a
\end{align*}
where $\sig_a$ is the automorphism defined by 
\[\sig_a(\zeta_n) = \zeta_n^a.\]
\end{thm}

\nl

\begin{cor}
Let $n = p_1^{a_1}p_2^{a_2}\cdots p_k^{a_k}$ be the decomposition of the positive integer $n$ into distinct prime powers. The the cyclotomic fields $\Q(\zeta_{p_i^{a_i}})$, $i = 1..k$ intersect only in the field $\Q$ and their composite is the cyclotomic field $\Q(\zeta_n)$. We have
\[\Gal(\Q(\zeta_n)/\Q)\cong \Gal(\Q(\zeta_{p_1^{a_1}})/\Q) \times \Gal(\Q(\zeta_{p_2^{a_2}})/\Q) \times \cdots \times \Gal(\Q(\zeta_{p_k^{a_k}})/\Q)\]
which under the isomorphism given in the previous theorem is the Chinese Remainder Theorem
\[(Z/n\Z)^\times \cong (\Z/p_1^{\al_1}\Z)^\times\times(\Z/p_2^{\al_2}\Z)^\times\times\cdots\times (\Z/p_k^{\al_k}\Z)^\times.\]
\end{cor}

\nl

\begin{defn}
The extension $K/F$ is called an \textit{\textbf{abelian}} extension if $K/F$ is Galois and $\Gal(K/F)$ is an abelian group.
\end{defn}

\nl

\begin{cor}
Let $G$ be any finite abelian group. Then there is a subfield $K$ of a cyclotomic field with $\Gal(K/\Q) \cong G$.
\end{cor}

\nl

\begin{thm}\textit{(Kronecker-Weber)}
Let $K$ be a finite abelian extension of $\Q$. Then $K$ is contained in a cyclotomic extension of $\Q$.
\end{thm}

\nl

\begin{prop}
\hl{The regular $n$-gon can be constructed by straightedge and compass if and only if $n = 2^kp_1\cdots p_r$ is the porduct of a power of 2 and distinct Fermat primes.}
\end{prop}

\nl

\textbf{Note:} Fermat primes are primes of the form $2^{2^n}+1$.












