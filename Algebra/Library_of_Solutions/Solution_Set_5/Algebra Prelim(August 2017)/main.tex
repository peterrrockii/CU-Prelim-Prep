\documentclass[11pt]{article}
 
\usepackage[margin=2cm]{geometry} 
\usepackage[oldstylenums]{kpfonts}
\usepackage{mathrsfs} %includes mathscript font
\usepackage{amsmath,amsthm,amssymb}
\usepackage[utf8]{inputenc}
\usepackage[english]{babel}
\usepackage{tikz}
\usepackage{mathtools}
\usetikzlibrary{matrix}
\usepackage{graphicx}
\usepackage{subcaption}
\usepackage{float}
\usepackage{stmaryrd}
\usepackage{hyperref}
\hypersetup{
    colorlinks=true,
    linkcolor=blue,
    filecolor=magenta,      
    urlcolor=cyan,
}
\usepackage{titling}

%setting legth for author/date
\setlength{\droptitle}{-5em}

%mapsto arrow with labeling on top
\makeatletter
\newcommand{\xMapsto}[2][]{\ext@arrow 0599{\Mapstofill@}{#1}{#2}}
\def\Mapstofill@{\arrowfill@{\Mapstochar\Relbar}\Relbar\Rightarrow}
\makeatother

%floor function
\newcommand{\floor}[1]{\lfloor #1 \rfloor}

%standard sets
\newcommand{\N}{\mathbb{N}}
\newcommand{\Z}{\mathbb{Z}}
\newcommand{\R}{\mathbb{R}}
\newcommand{\C}{\mathbb{C}}
\newcommand{\Hq}{\mathbb{H}}
\newcommand{\F}{\mathbb{F}}

%Quaternions
\newcommand{\1}{\textbf{1}}
\newcommand{\bi}{\textbf{i}}
\newcommand{\bj}{\textbf{j}}
\newcommand{\bk}{\textbf{k}}

 
\usepackage{amsthm}

\newtheorem{stheorem}{Theorem}[section]
\newtheorem{scor}{Corollary}[section]
\newtheorem{slemma}{Lemma}[section]
\newtheorem{sprop}{Proposition}[section]
\newtheorem{sconj}{Conjecture}[section]
\newtheorem{sex}{Exercise}[section]
\newtheorem{sdefi}{Definition}[section]
\newtheorem{sexa}{Example}[section]
\newtheorem{sprob}{Problem}[section]


\newtheorem{theorem}{Theorem}
\newtheorem{cor}{Corollary}
\newtheorem{lemma}{Lemma}
\newtheorem{prop}{Proposition}
\newtheorem{conj}{Conjecture}
\newtheorem{ex}{Exercise}
\newtheorem{defi}{Definition}
\newtheorem{exa}{Example}
\newtheorem{prob}{Problem}


\newtheorem*{theorem*}{Theorem}
\newtheorem*{cor*}{Corollary}
\newtheorem*{lemma*}{Lemma}
\newtheorem*{prop*}{Proposition}
\newtheorem*{conj*}{Conjecture}
\newtheorem*{ex*}{Exercise}
\newtheorem*{defi*}{Definition}
\newtheorem*{exa*}{Example}

\newtheorem*{sol*}{\textit{Solution}}

%transverse sign as in Guillemin & Pollack
\newcommand{\transv}{\mathrel{\text{\tpitchfork}}}
\makeatletter
\newcommand{\tpitchfork}{%
  \vbox{
    \baselineskip\z@skip
    \lineskip-.52ex
    \lineskiplimit\maxdimen
    \m@th
    \ialign{##\crcr\hidewidth\smash{$-$}\hidewidth\crcr$\pitchfork$\crcr}
  }%
}
\makeatother



\begin{document}

\title{\LARGE \textbf{CU Boulder: \textit{Algebra} %subject of exam
Prelim \\ \textit{August 2017}} %date exam was administered
\vspace{-.75cm}}% vspace sets margin between title and author
\author{Juan Moreno
} 
\date{\vspace{-0.45cm}April 2019} % can input date if desired, vspace sets margin between author and date

 
\maketitle

%remove abstract title
\renewcommand{\abstractname}{\vspace{-\baselineskip}}
%insert description/ abstract
\begin{abstract}
\noindent These are my solutions to the questions on the CU Boulder \textit{Algebra} preliminary exam from \textit{August 2017} found  \href{http://math.colorado.edu/documents/graduate/prelim/Algebra_Aug_2017.pdf}{here}. I worked on these solutions over the summer of 2019 in preparation for the preliminary exam in the Fall 2019. Please send any questions, comments, or corrections to \href{mailto: juan.moreno-1@boulder.edu}{juan.moreno-1@boulder.edu.} \\
\end{abstract}

\begin{prob}
Assume that $G$ is an infinite nonabelian group whose proper subgroups are finite. Show that every proper normal subgroup of $G$ is contained in the center of $G$. Explain why $G/Z(G)$ is an infinite simple group whose proper subgroups are finite. 

\begin{proof}
Let $N\trianglelefteq G$ be a proper normal subgroup of $G$. Then $G$ acts on $N$ by conjugation, giving rise to a homomorphism $\varphi:G\rightarrow S_n$, where $n = |N|$. The kernel of this map must then also be a normal subgroup. This leaves us two options, either ker$\varphi$ is finite or ker$\varphi = G$. In the first case, however, we would have the infinite quotient $G/\textnormal{ker}\varphi$ being isomorphic to a subgroup of the finite group $S_n$, a contradiction. Hence ker$\varphi = G$ so that action of $G$ on $N$ by conjugation is trivial, implying $N$ lies in the center of $G$. The last statement follows mostly from the lattice isomorphism theorem since any normal subgroup of $G/Z(G)$ corresponds to a normal subgroup containing $Z(G)$, but as we have shown, all proper normal subgroups are contained in $Z(G)$. Thus the only normal subgroups of $G/Z(G)$ are the trivial subgroup and the entire group, hence $G/Z(G)$ is simple. Similarly, any proper subgroup of $G/Z(G)$ is isomorphic to the quotient of a proper subgroup of $G$ containing $Z(G)$ by $Z(G)$, which must be finite by assumption. It is infinite since $Z(G)$ is normal in $G$ and since $G$ is nonabelian, it is proper and thus finite, implying $G/Z(G)$ is infinite. 
\end{proof}
\end{prob}


\begin{prob}
Suppose the alternating group $A_4$ acts transitively on a set $X$. What are the possible sizes of $X$. 

\begin{sol*}

\textnormal{ For a group $G$, define a transitive $G$-set to be a set $X$ with a transitive action by $G$. Define an isomorphism of $G$-sets $X$ and $Y$ to be a bijective map of sets $f:X\rightarrow Y$ which preserves the $G$-action, i.e. $f(g\cdot x) = g \cdot f(x)$ for all $g\in G$. For $x\in X$, let $G_x$ be the stabilizer of $x$ under the $G$-action. We prove that any transitive $G$-set $X$ is isomorphic to the set of cosets $G/G_x$ for any $x\in X$. Simply pick any $x\in X$, and define the map $\varphi:G\rightarrow X$ by $\varphi(g) = g\cdot x$. Evidently, this map factors through the map $\pi :G\rightarrow G/G_x$ since $G_x\cdot x = x$. So we have the following commutative diagram
}

\begin{center}
\begin{tikzpicture}[node distance = 1.6cm, auto]
\node (G) {$G$};
\node (G') [below of= G] {$G/G_x$};
\node (X) [right of = G'] {$X$};
\draw[->] (G) -- (G') node[midway, left] {$\pi$};
\draw[->] (G) -- (X) node[midway] {$\varphi$};
\draw[->,dashed] (G') -- (X) node[midway,below] {$\overline{\varphi}$};

\end{tikzpicture}
\end{center}
\textnormal{ We claim that the induced map $\overline{\varphi}$ is a $G$-set isomorphism. To see this, simply note that $|G/G_x| = |X|$ and compute for any $g\in G$, $\overline{\varphi}(g\cdot hG_x) = \overline{\varphi}((gh)G_x) = (gh)G_x\cdot x = (gh)\cdot x = g\cdot(h\cdot x) = g\cdot(hG_x\cdot x) = g\overline{\varphi}(hG_x)$. This proves the result.
}

\textnormal{Now consider the case $G= A_4$. By the above, any set $X$ on which $A_4$ acts transitively, is isomorphic as an $A_4$-set to some set of cosets of $A_4$. Since $A_4$ has subgroups of order $1, 2, 3, 4,$ and $12$, the possible sizes of sets of cosets and hence sets on which $G$ acts transitively are $12, 6, 4, 3,$ and $1$. 
}

\end{sol*}
\end{prob}

\begin{prob}
Let $A$ be an integral domain containing the field  $\F$ as a subring. This makes $A$ a vector space over $\F$. Show that if $A$ is finite dimensional over $\F$ then $A$ is a field. Show that $A$ need not be a field if it is not finite dimensional over $\F$.  

\begin{proof}
Assume $A$ is finite dimensional over $\F$. Take any nonzero $r\in A$. Consider the set of powers of $r$, $\{r^k\}_{k=0}^\infty$. If this set is finite, then we must have $r^k = r^{k'}$ for some $k,k'$. Using the cancellation property of multiplication in integral domains we have that $r^l = 1$ for some $l$ so that $r$ is a unit in $A$ with inverse $r^{l-1}$. If, on the other hand the set is infinite, by finite dimensionality of $A$ over $\F$, we have that there exists some $n\in\N$ and $c_0,c_1,...,c_n\in\F$ not all zero such that $\sum_{i = 0}^n c_ir^i = 0$. Notice that if $k$ is the minimal number such that $c_k\neq = 0$ then we may write $\sum_{i = k}^n c_i r^i = r^k\sum_{i = 0}c_i r^{i-k} = 0$, and since $A$ is an integral domain and $r\neq 0$, we have $\sum_{i = k}^n c_i r^{i-k}$. Therefore, we may assume $c_0\neq 0$. Let $b_i = \frac{c_i}{c_0}$ so that, in particular, $b_0 = 1$. Then \[\sum_{i = 0}^n c_i r^i = 0 \implies \sum_{i = 0}^n b_i r^i = 0 \implies 1 = \sum_{i = 1}^n (-b_i) r^i.\] Since the left side of the final expression above must be nonzero ($1\neq 0$ in a nontrivial ring) and the indexing begins at $i = 1$, we may factor out at least one factor of $r$ and write \[r\sum_{i = 0}^n(-b_i)r^i = 1,\] implying $r$ has an inverse in $A$. 
\end{proof}
\end{prob}

\begin{prob}
You are given that $G$ is a group for which there exists a surjective homomorphism $\alpha:\Z^n\rightarrow G$ and an injective homomorphism $\beta :\Z^n\rightarrow G$. What are the possible isomorphism classes of $G$? 

\begin{sol*}
\textnormal{Since we have a surjective homomorphism from the abelian group  $\Z^n$ onto $G$, we must have that $G$ is abelian. Further, since $\Z^n$ has $n$ generators, and $\alpha$ is determined by the images of these generators, the fact that $\alpha$ is surjective implies that $G$ has at most $n$ generators. By the classification of finitely generated abelian groups, we have that \[G\cong \Z^k\times\Z/(a_1)\times\cdots\times\Z/(a_l),\] for some $k,l\in\N$ such that $k + l \leq n$, and $a_i\in\Z$. Here $k$ is the free rank of $G$. Now the existence of the injective map $\beta$ from $\Z^n$ into $G$, implies that $G$ has a subgroup isomorphic to $\Z^n$, implying that the free rank of $G$ is at least $n$. It follows that $k = n$ and $l = 0$ so that $G\cong\Z^n$. 
}
\end{sol*}
\end{prob}

\begin{prob}
Consider the following three rings \[\F_3[x]/(x^2 + 1), \quad \F_3[x](x^2 + 2), \quad \textnormal{ and } \F_3[x]/(x^2 +2x + 2),\] where $\F_3$ is the field with $3$ elements.

\noindent (a) Show that each of these rings is a product of fields and say which fields are involved.

\begin{sol*}
\textnormal{Let $p_1(x) = x^2 + 1, p_2(x) = x^2 + 2, p_3(x) = x^2 + 2x + 2$ and $K_i = \F_3[x]/(p_i(x))$. Since these polynomials are all of degree $2$ it is trivial to check by finding roots that $p_1(x)$ and $p_3(x)$ are irreducible and $p_2(x) = (x + 1)(x + 2)$. Since $\F_3$ is a field, $\F_3[x]$ is a PID so that both $p_1(x)$ and $p_3(x)$ must be prime hence generate maximal ideals. It follows that $K_1$ and $K_3$ are fields. Further, as sets each of these are of the form $\{a + b\bar{x} | a,b\in\F_3\}$, where $\bar{x}$ denotes the image of $x$ in $K_i$. These are both finite fields of the same order, namely $9$. Thus, $K_1 \cong K_3\cong \F_9$.  As for $p_2(x)$, since $2(x+1) + (x + 2) = 1$, as ideals we have $(x +1 ) + (x +2) = \F_3[x]$. Moreover, since $x+1$ and $x+2$ are irreducible in $\F_3[x],$ $(x+1)\bigcap (x+2)$ is notrivial only if $(x+1) = (x +2)$ since this intersection would be generated by a greatest common divisor of $x + 1$ and $x + 2$. This can only be the case if $x+1$ and $x+2$ differ by a unit in $\F_3[x]$, which is not the case since they are not multiples of one another as can easily be checked. Thus, by the Chinese Remainder Theorem \[K_2[x] = \F_3[x]/(x^2 + 2)\cong \F_3[x]/(x + 1)\times \F_3[x]/(x+2)\cong \F_3\times\F_3.\]
}
\end{sol*}


\noindent (b) For each pair of isomorphic rings in the list, provide an explicit isomorphism.

\textnormal{To exhibit an explicit isomorphism between the fields $K_1$ and $K_3$, let $\alpha$ denote the image of $x$ under the projection $\F_3[x]\rightarrow K_1$ and $\beta $ the image of $x$ under the projection $\F_3[x]\rightarrow K_2$. Then $\alpha^2 = 2$ and $\beta^2 = \beta + 1\implies (\beta + 1)^2 = \beta^2 + 2\beta + 1 = 2$. We can then define a map $\varphi:K_1\rightarrow K_3$ by requiring it restrict to the identity on $\F_3$ and map $\alpha\mapsto \beta + 1$. To see that this is a field homomorphism, take any $a+b\alpha, c + d\alpha\in K_1$ and compute \[\varphi((a + b\alpha)(c+d\alpha)) = \varphi((ac + 2bd) + (ad + bc)\alpha) = (ac + 2bd)+(ad + bc)(\beta + 1), \] and  \[\varphi(a + b\alpha)\varphi(c + d\alpha) = (a + b(\beta + 1))(c + d(\beta + 1)) = (ac + bd(\beta + 1)^2) + (ad + bc)(\beta +1) = (ac + 2bd) + (ad + bc)(\beta + 1). \] The additive property of $\varphi$ follows simply from its definition, so $\varphi$ is indeed a field homomorphism. It is also evidently nontrivial and so it must be an isomorphism onto its image. Since these fields have the same cardinality, we have that $\varphi$ is an explicit isomorphism between the two fields $K_1$ and $K_3$.
}
\end{prob}

\begin{prob}
Let $p\geq 5$ be a prime number and let $L$ be the splitting field of $x^p -1$ over $\mathbb{Q}$. 

\noindent (a) Find explicit generators for the Galois group Gal$(L/\mathbb{Q})$ and explain why your answer is correct. What is the structure of this group?

\begin{sol*}
\textnormal{We view $\mathbb{Q}$ as a subfield of $\C$ as usual. Then $\alpha_k = e^{2\pi k i/p}, k = 0,1,...,p-1$ are the roots of $p(x) = x^2 -1$ in $\C$. Notice that if $\alpha_k\in\mathbb{Q}$ then $2\pi k /p  = \pi l$ for some $l\in\Z$, implying $2 k / p \in\Z $, however, this cannot be unless $k = 0$ since $k < p$ and $p$ is an odd prime. Thus, the only root of $p(x)$ in $\mathbb{Q}$ is $\alpha_0 = 1$. Moreover, note that $\alpha_k = \alpha_1^k$ for all $k = 0,1,...,p-1$. Hence $L = \mathbb{Q}(\alpha_1)\cong \mathbb{Q}[x]/(q(x))$ where $q(x) = \frac{x^p-1}{x-1}$. We now have that $[L:\mathbb{Q}] = |\textnormal{Gal}(L/\mathbb{Q})| = p-1$ and that this Galois group must act transitively on the roots of $q(x)$ since it is irreducible and $L$ is its splitting field. Let $\sigma_k:L\rightarrow L$ be the automorphism which fixes $\mathbb{Q}$ and maps $\alpha_1\mapsto \alpha_k$, for $k = 1,2,...,p-1$. We can quickly investigate how these automorphisms relate \[\sigma_l\circ\sigma_k(\alpha_1) = \sigma_l(\alpha_k) = \sigma_l(\alpha_1^k) = \sigma_l(\alpha_1)^k = \alpha_l^k = \alpha_1^{lk} = \alpha_{lk} = \sigma_{lk}(\alpha_1).\] It follows that Gal$(L/\mathbb{Q})\cong Z_{p-1}$ and is generated by any $\sigma_k$ such that $k$ is a generator of $\Z_p^\times$. 
}
\end{sol*}

\noindent (b) Use (a) to find explicit generators for a subfield $K$ of $L$ such that $[L:K] =2$ and explain why your answer is correct. 

\begin{sol*}
\textnormal{In part (a) we found that the Galois group of $K$ over $\mathbb{Q}$ is cyclic of order $p-1$. By the fundamental theorem of Galois Theory, to find a subfield of $L$ of index $2$ is equivalent to finding a subgroup of the Galois group of order $2$. Such a subgroup can be found simply by noting that the automorphism of complex conjugation on $\C$ restricts to the identity on $\mathbb{Q}$ and the nontrivial automorphism $\sigma_{p-1}:\alpha_1\mapsto \alpha_{p-1}$ of $L$. Since complex conjugation is a transformation of order $2$, $\sigma_{p-1}$ has order $2$ in Gal$(L/\mathbb{Q})$ and so we have found a subgroup of order $2$, $\langle\sigma_{p-1}\rangle$. To find its corresponding fixed field, note that the elements \[\theta_1 = \alpha_1 + \sigma_{p-1}\alpha_1 = 2\textnormal{Re}(\alpha_1),\]\[\theta_2 = \alpha_2 + \sigma_{p-1}\alpha_2 = 2\textnormal{Re}(\alpha_2),\] \[\vdots\] \[\theta_{\frac{p-1}{2}} =  \alpha_{\frac{p-1}{2}} + \sigma_{p-1}\alpha_{\frac{p-1}{2}} = 2\textnormal{Re}(\alpha_{\frac{p-1}{2}}),\] are each distinct and fixed by $\sigma_{p-1}$. Moreover, since Re$(\alpha_k) = \cos(2\pi k/ p)$
}
\end{sol*}
\end{prob}

\end{document}
