\documentclass[11pt]{article}
 
\usepackage[margin=2cm]{geometry} 
\usepackage[oldstylenums]{kpfonts}
\usepackage{mathrsfs} %includes mathscript font
\usepackage{amsmath,amsthm,amssymb}
\usepackage[utf8]{inputenc}
\usepackage[english]{babel}
\usepackage{tikz}
\usepackage{mathtools}
\usetikzlibrary{matrix}
\usepackage{graphicx}
\usepackage{subcaption}
\usepackage{float}
\usepackage{stmaryrd}
\usepackage{hyperref}
\hypersetup{
    colorlinks=true,
    linkcolor=blue,
    filecolor=magenta,      
    urlcolor=cyan,
}
\usepackage{titling}

%setting legth for author/date
\setlength{\droptitle}{-5em}

%mapsto arrow with labeling on top
\makeatletter
\newcommand{\xMapsto}[2][]{\ext@arrow 0599{\Mapstofill@}{#1}{#2}}
\def\Mapstofill@{\arrowfill@{\Mapstochar\Relbar}\Relbar\Rightarrow}
\makeatother

%floor function
\newcommand{\floor}[1]{\lfloor #1 \rfloor}

%standard sets
\newcommand{\N}{\mathbb{N}}
\newcommand{\Z}{\mathbb{Z}}
\newcommand{\R}{\mathbb{R}}
\newcommand{\C}{\mathbb{C}}
\newcommand{\Hq}{\mathbb{H}}
\newcommand{\F}{\mathbb{F}}

%Quaternions
\newcommand{\1}{\textbf{1}}
\newcommand{\bi}{\textbf{i}}
\newcommand{\bj}{\textbf{j}}
\newcommand{\bk}{\textbf{k}}

 
\usepackage{amsthm}

\newtheorem{stheorem}{Theorem}[section]
\newtheorem{scor}{Corollary}[section]
\newtheorem{slemma}{Lemma}[section]
\newtheorem{sprop}{Proposition}[section]
\newtheorem{sconj}{Conjecture}[section]
\newtheorem{sex}{Exercise}[section]
\newtheorem{sdefi}{Definition}[section]
\newtheorem{sexa}{Example}[section]
\newtheorem{sprob}{Problem}[section]


\newtheorem{theorem}{Theorem}
\newtheorem{cor}{Corollary}
\newtheorem{lemma}{Lemma}
\newtheorem{prop}{Proposition}
\newtheorem{conj}{Conjecture}
\newtheorem{ex}{Exercise}
\newtheorem{defi}{Definition}
\newtheorem{exa}{Example}
\newtheorem{prob}{Problem}


\newtheorem*{theorem*}{Theorem}
\newtheorem*{cor*}{Corollary}
\newtheorem*{lemma*}{Lemma}
\newtheorem*{prop*}{Proposition}
\newtheorem*{conj*}{Conjecture}
\newtheorem*{ex*}{Exercise}
\newtheorem*{defi*}{Definition}
\newtheorem*{exa*}{Example}

\newtheorem*{sol*}{\textit{Solution}}

%transverse sign as in Guillemin & Pollack
\newcommand{\transv}{\mathrel{\text{\tpitchfork}}}
\makeatletter
\newcommand{\tpitchfork}{%
  \vbox{
    \baselineskip\z@skip
    \lineskip-.52ex
    \lineskiplimit\maxdimen
    \m@th
    \ialign{##\crcr\hidewidth\smash{$-$}\hidewidth\crcr$\pitchfork$\crcr}
  }%
}
\makeatother



\begin{document}

\title{\LARGE \textbf{CU Boulder: \textit{Algebra} %subject of exam
Prelim \\ \textit{January 2018}} %date exam was administered
\vspace{-.75cm}}% vspace sets margin between title and author
\author{Juan Moreno
} 
\date{\vspace{-0.45cm}April 2019} % can input date if desired, vspace sets margin between author and date

 
\maketitle

%remove abstract title
\renewcommand{\abstractname}{\vspace{-\baselineskip}}
%insert description/ abstract
\begin{abstract}
\noindent These are my solutions to the questions on the CU Boulder \textit{Algebra} preliminary exam from \textit{January 2018} found  \href{http://math.colorado.edu/documents/graduate/prelim/Algebra_Jan_2018.pdf}{here}. I worked on these solutions over the summer of 2019 in preparation for the preliminary exam in the Fall 2019. Please send any questions, comments, or corrections to \href{mailto: juan.moreno-1@boulder.edu}{juan.moreno-1@boulder.edu.} \\
\end{abstract}


\begin{prob}
Let $G$ be the symmetric group $S_5$ and $P$ a Sylow $5$-subgroup of $G$. 

\noindent (i) Show that the normalizer $N_G(P)$ has order $20$. 

\begin{proof}
By Sylow's Theorem, $|\textnormal{Syl}_5(G)| \equiv 1(\textnormal{mod}5)$ and $|\textnormal{Syl}_5(G)|$ divides $\frac{120}{5} = 24$. Thus $|\textnormal{Syl}_5(G) =  1$ or $6$. However, there are $24$  $5$-cycles in $S_5$ so in fact $|\textnormal{Syl}_5(G)| = 6$. Sylow's Theorem also states that $G$ acts on the set $\textnormal{Syl}_5(G)$ by conjugation. Since $5$-cycles constitute an equivalence class in $G$, this action is transitive. Observing that the stabilizer of $P$ under this action is $N_G(P)$, orbit-stabilizer theorem then implies that \[|G| = |N_G(P)|\cdot | \textnormal{Syl}_5(G)| \implies |N_G(P)| = \frac{120}{6} = 20.\] 
\end{proof}

\noindent (ii) In the special case when $P$ contains the $5$-cycle $(12345)$, find a set of generators for $N_G(P)$. 

\begin{sol*}
\textnormal{Clearly $P = \langle (12345)\rangle\leq N_G(P)$. Now  $(12345)^4 = (12345)^{-1} = (15432) = ((25)(34))(12345)((25)(34))^{-1}$. It follows that $\sigma = (12345),\tau = (25)(34)\in N_G(P)$ and these elements satisfy the relations \[\sigma^5 = \tau^2 = 1,\quad  \sigma\tau = \tau\sigma^{-1},\] so that these elements generate a subgroup of $N_G(P)$ isomorphic to the dihedral group $D_{20}$. Since this group has order $20$, we must have $N_G(P) = \langle\sigma,\tau\rangle \cong D_{20}$ 
}
\end{sol*}
\end{prob}

\begin{prob}
Let $G$ be a group and $Z(G)$ be the center of the group. An automorphism $\alpha\in\textnormal{Aut}(G)$ is said to be central if for all $x\in G$ we have $x^{-1}\alpha(x)\in Z(G)$. Show that the central automorphisms form a normal subgroup $N$ of $\textnormal{Aut}(G)$.

\begin{proof}
The identity automorphism is clearly central. As for inverses, in $\alpha\in\textnormal{Aut}(G)$ is central then for all $x\in G$, $x\alpha(x^{-1})\in Z(G)$ and since automorphisms must preserve the center, $\alpha^{-1}(x\alpha(x^{-1})) = \alpha^{-1}(x) x^{-1}\in Z(G)$, so that $\alpha^{-1}(x)x^{-1} = x^{-1}(\alpha^{-1}(x) x^{-1})x =x^{-1}\alpha^{-1}(x) \in Z(G)$. So the inverse of a central automorphism is also central. To see that the central automorphisms are closed under composition, take two central automorphisms $\alpha, \beta$ and compute \[x^{-1} \alpha\circ\beta (x) = x^{-1}\alpha(\beta(x)) = x^{-1}\beta(x)\beta(x)^{-1}\alpha(\beta(x)). \] Since both $\beta$ and $\alpha$ are central, $x^{-1}\beta(x), \beta(x)^{-1}\alpha(\beta(x))\in Z(G)$. That the central automorphisms are closed under composition then follows from the fact that $Z(G)$, being a subgroup of $G$, is closed under multiplication. Lastly, to see that this group is indeed normal we simply compute the following for any central $\alpha$, any $\beta\in\textnormal{Aut}(G)$ and all $x\in G$: \[x^{-1}(\beta^{-1}\circ\alpha\circ\beta(x)) = \beta^{-1}(\beta(x^{-1})\cdot\alpha(\beta(x))) = \beta^{-1}(\beta(x)^{-1}\cdot\alpha(\beta(x))).\] Since $\alpha$ is central, $\beta(x)^{-1}\cdot\alpha(\beta(x))\in Z(G)$ and since $\beta$, being an automorphism of $G$, preserves the center, we have $\beta^{-1}(\beta(x)^{-1}\cdot\alpha(\beta(x)))\in Z(G)$. Thus $\beta^{-1}\circ\alpha\circ\beta$ is central, implying the subgroup of $\textnormal{Aut}(G)$ of central automorphisms is indeed a normal sugroup. 

\end{proof}
\end{prob}

\begin{prob}
Let $k$ be a field and $R$ the subring of $k(x)$ generated by $k[x]$ and $1/x$. For a typical nonzero element $p(x) = \sum_{i = -M}^N a_i x^i$ of $R$, define \[H(p(x)) = max(\{ i\in\Z| a_i\neq 0\})\quad \textnormal{ and }\quad L(p(x)) = min(\{i\in\Z | a_i\neq 0\}).\] Show that $R$ is a Euclidean domain with Euclidean norm given by $N(p(x)) = H(p(x)) - L(p(x))$ and $N(0) = 0$.

\begin{proof}
First we note that $R$ is an integral domain since it is a subring of the field $k(x)$. It remains to show that the Division algorithm holds for any two $p(x), q(x)\in R$ with $q(x)\neq 0$. Suppose first that $L(p(x)), L(q(x))\geq 0$. Then in fact $p(x), q(x) \in k[x]$ and using the standard division algorithm we may write \[p(x) = q(x)\cdot b(x)  + r(x),\] for some $b(x),r(x)\in k[x]$ with either $r(x) = 0$ or deg$(r(x)) <$ deg$(q(x))$. Note that in this case deg$(r(x)) = H((r(x))$. 
\end{proof}
\end{prob}

\begin{prob}
Let $F$ be a field of arbitrary characteristic. Show that any two elements of order $2$ in the special linear group $SL_2(F)$ are conjugate in $GL_2(F)$. Find a necessary and sufficient condition on $F$ for $SL_2(F)$ to have a unique element of order $2$.

\begin{sol*}
\textnormal{Let $A\in SL_2(F)$ be an element of order $2$. Then $A$ satisfies the equation $p(A) = 0$, where $p(x) = x^2 - 1$. Now recall that the conjugacy class of a matrix is completely determined by a list of invariant factors $a_0(x), a_1(x),...,a_n(x)$ such that $a_i(x)$ divides $a_{i+1}(x)$ for all $i = 0,..., n-1$, and the product $\prod_{i = 0}^n a_i(x)$ is the characteristic polynomial of the matrix, $c_A(x)$. Since in this case $c_A(x)$ is of degree $2$, there are only two possible lists of invariant factors, namely $L1 = \{c_A(x)\}$ or $L2 = \{x - a, x - a\}$, for some $a\in F$ such that $(x-a)^2 = c_A(x)$. The $L2$ case implies the only root of $c_A(x)$, that is, the unique eigenvalue of $A$ must be $a \in F$. Since $A\in SL_2(F)$ det$A = a^2 = 1$ so either $a = 1$ or $a = -1$. If $a = 1$ then $A$ is simply the identity matrix, which we do not actually consider an element of order $2$. If $a = -1$, the rational canonical form of $A$ is then $\begin{pmatrix}-1 & 0 \\ 0 & -1 \end{pmatrix}$. Note that this matrix is still the identity if char$F = 2$. Now consider the $L1$ case. Let $c_A(x) = x^2 + bx + c$. The rational canonical form of $A$ is then $\begin{pmatrix} 0 & -c \\ 1 & -b\end{pmatrix}$. Since det$A = 1$, we must have $c = 1$, but then \[\begin{pmatrix} 0 & -1 \\ 1 & -b\end{pmatrix}^2 = \begin{pmatrix}-1 & b \\ -b & b^2 - 1 \end{pmatrix}.\] The only way this matrix could equal the $2\times 2$ identity is if char$F = 2$ and $b = 0$. The rational canonical form is then $\begin{pmatrix}0 & 1 \\ 1 & 0 \end{pmatrix}$. Thus, if char$F\neq 2$ the unique rational canonical form representing matrices of order $2$ and determinant $1$ in $GL_2(F)$ is $\begin{pmatrix} -1 & 0 \\ 0 & -1\end{pmatrix}$ and if char$F = 2 $ the representing matrix is $\begin{pmatrix} 0 & 1 \\ 1 & 0\end{pmatrix}$.
}

\noindent\textnormal{Since the rational canonical forms derived above are examples of matrices in $SL_2(F)$ of order $2$, there is a unique such matrix in $SL_2(F)$ if and only if these matrices constitute their own conjugacy class, i.e. if and only if these matrices are in the center of $GL_2(F)$. Suppose first that char$F \neq 2$. Then the matrix in question is scalar and hence lies in the center of $GL_2(F)$ so it must be the unique element of $SL_2(F)$ of order $2$. Now if char$F = 2$ then we find that  \[\begin{pmatrix} 1 & 1 \\ 0 & 1 \end{pmatrix}\cdot\begin{pmatrix} 0 & 1 \\ 1 & 0\end{pmatrix} = \begin{pmatrix} 1 & 1 \\ 1 & 0\end{pmatrix}, \] and \[\begin{pmatrix} 0 & 1 \\ 1 & 0 \end{pmatrix} \cdot \begin{pmatrix} 1 & 1 \\ 0 & 1\end{pmatrix} = \begin{pmatrix} 0 & 1 \\ 1 & 1 \end{pmatrix}\] so the matrix in question does not lie in the center. Thus, a necessary and sufficient condition for $SL_2(F)$ to have a unique element of order $2$ is for char$F\neq 2$. 
}

\end{sol*}
\end{prob}

\begin{prob}
Let $p$ be a prime, $\F_p$ be the field with $p$ elements, and let $t$ be an indeterminate. Let $F = \F_p(t)$ be the field of fractions of the polynomial ring $\F_p[t]$. 

\noindent (i) Show that $g(x) = x^p - x + t$ is separable over $F$. 

\begin{proof}
The derivative of $g$ is $D_x(g(x)) = -1$. This polynomial has no roots, hence $g(x)$ is relatively prime to its derivative so it must be separable.
\end{proof}

\noindent (ii) Show that if $\alpha$ is a root of $g$ then $\alpha + 1$ is also a root. Deduce that the roots of $g$ are precisely those of the form $\alpha + b$ for $b\in\F_p$. 

\begin{proof}
If $\alpha$ is a root of $g$ then \[(\alpha + 1)^p - (\alpha + 1) + t = \alpha^p + 1 - (\alpha + 1) + t = \alpha^p - \alpha + t = g(\alpha) = 0.\] Thus $\alpha + 1$ is also a root of $g$. Replacing $\alpha$ with $\alpha + 1$ and repeating the above computation $p-1$ times shows that $\alpha, \alpha + 1,...,\alpha + (p-1)$ are all roots of $g$. Since $g$ is a degree $p$ polynomial, it can have at most $p$ roots, therefore there are all the roots of $g$. 

\end{proof}

\noindent (iii) Show that $g$ has no roots in $F$.  

\begin{proof}
If $a\in \F_p$ were a root, then the set of roots of $g$ is $\{a + b| b\in\F_p\} = \F_p$. To see this simply take $\alpha = a + (-a) = 1$ and proceed as in part $b$. We then have that $g$ factors in $F[x]$ as \[g(x) = (x - 1)(x - 2)\cdots (x-(p-1)).\] However, the constant term of $g$ would then be $\prod_{b\in\F_p}b \neq t$, a contradiction. 
\end{proof}


\noindent (iv) Find the Galois group of $g$ over $F$. 

\begin{sol*}
\textnormal{By part (ii), $g$ splits in $F(\alpha)\cong F[x]/(g(x))$ where $\alpha$ is any root of $g$. Part (iii) shows that $F(\alpha)\neq F$ so that $[F(\alpha): F] = p$. It follows that this must be the splitting field of $g$. Since $g$ is separable, so is $F(\alpha)$ hence $|\textnormal{Gal}(F(\alpha)/F)| = [F(\alpha):F] = p$. Since $p$ is a prime, the only group of order $p$ is the cyclic group of order $p$, thus $\textnormal{Gal}(F(\alpha)/F)\cong Z_p$. 
}
\end{sol*}
\end{prob}

\begin{prob}
Let $f(x)$ be a monic polynomial of degree $n> 0$ over a field $K$ and let $\Delta(f)$ denote its discriminant. Let $g(x) = f(x^2)$. You may assume without proof that $\Delta(g) = \Delta(f)^2(-4)^n f(0)$.

\noindent (i) Let $f(x) = x^2 + 3x + 1$ so that $g(x) = x^4 + 3x^2 + 1$. Show that $g$ is irreducible over $\mathbb{Q}$. 

\begin{proof}
Viewing $\mathbb{Q}$ as a subfield of $\C$, we know that the roots of $f(x)$ are \[r_\pm = \frac{-3 \pm\sqrt{5}}{2}.\] We then have that $g(\sqrt{r_\pm}) = f(r_\pm) = 0$ so that $\sqrt{r_\pm}$ are both roots of $g$ in $\C$. Further, since $g(-x) = g(x)$, we have that $-\sqrt{r_\pm}$ are also roots of $g$ in $\C$. This accounts for $4$ roots of $g$ in $\C$ and since $g$ has degree $4$, these must be all of its roots. Since none of these roots lie in $\mathbb{Q}$, $g$ does not factor over $\mathbb{Q}$ into a product of $4$ degree $1$ polynomial or a product of a degree $1$ polynomial and a degree $3$ polynomial. The only possibilities left are either $g$ is the product of two irreducible quadratics or $g$ is irreducible. That the former case does not hold and can be checked simply looking at pairwise products of the factors $\{(x - \sqrt{r_\pm}), (x + \sqrt{r_\pm})\}$:  \[(x - \sqrt{r_\pm})(x + \sqrt{r_\pm}) = x^2 - r_\pm\notin\mathbb{Q}[x], \] \[(x - \sqrt{r_+})(x +\sqrt{r_-}) = x^2 + (\sqrt{r_-} - \sqrt{r_+})x - \sqrt{r_+ r_-}\notin\mathbb{Q}[x], \] \[(x - \sqrt{r_+})(x - \sqrt{r_-}) = x^2 - (\sqrt{r_+} + \sqrt{r_-})x + \sqrt{r_+ r_-}\notin\mathbb{Q}[x].\]
\end{proof}

\noindent (ii) To which familiar group is the Galois group of $g$ over $\mathbb{Q}$ isomorphic?

\begin{sol*}
\textnormal{The Galois group of $g$ must be a subgroup of $S_4$ since it permutes the roots of $g$. Further, since $\Delta(g) = \Delta(f)^2(-4)^2 = 25\cdot16 \in\mathbb{Q} $, its discriminant must be fixed by the Galois group and so this group must in fact lie in $A_4$. Let $\alpha_1 = \sqrt{r_+}, \alpha_2 = -\sqrt{r_+},\alpha_3 = \sqrt{r_-},\alpha_4 = -\sqrt{r_-}$ and consider the following elements of $K$ \[\theta_1 = (\alpha_1 + \alpha_2)(\alpha_3 + \alpha_4) = 0\] \[\theta_2 = (\alpha_1 + \alpha_3)(\alpha_2 + \alpha_4) = -(\sqrt{r_+} + \sqrt{r_-})^2 = 1\] \[\theta_3 = (\alpha_1 + \alpha_4)(\alpha_2 + \alpha_3) = -(\sqrt{r_+} - \sqrt{r_-})^2 = -1. \] These elements, as defined, are permuted by the Gal$(K/\mathbb{Q})$. We view this group as a subgroup of $S_4$ via the action on the $\alpha_i$. The stabilizer of the $\theta_i$ is the Klein 4-group $V = \{1, (12)(34), (13)(24), (14)(23)\}$. Since $g$ is irreducible, $[K:\mathbb{Q}] = |\textnormal{Gal}(K/\mathbb{Q})|$ is at least $4$. Thus, Gal$(K/\mathbb{Q}) = V$.}

\end{sol*}
\end{prob}


\end{document}
