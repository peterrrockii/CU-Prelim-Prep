\documentclass[12pt]{extarticle}

\usepackage{geometry}
\geometry{a4paper, portrait}
\usepackage[utf8]{inputenc}
\usepackage{setspace}
\usepackage{mathtools}
\usepackage{amsfonts}
\usepackage{amsmath}
\usepackage{amssymb}
\usepackage[shortlabels]{enumitem}
\usepackage{physics}
\usepackage{textcomp}
\usepackage{faktor}
\usepackage{tikz-cd}
\usepackage{hyperref}
\hypersetup{
    colorlinks=true,
    linkcolor=blue,
    filecolor=magenta,      
    urlcolor=cyan,
    }

\DeclareMathOperator{\im}{Im}
\DeclareMathOperator{\Gal}{Gal}

\newcommand{\ZZ}{\mathbb{Z}}
\newcommand{\QQ}{\mathbb{Q}}
\newcommand{\NN}{\mathbb{N}}

\newcommand{\set}[1]{\left\{ #1  \right\}}
\linespread{1.2}
\setlength{\parskip}{1em}

\title{Algebra prelim solutions August 2015}
\author{Calum Shearer}
\begin{document}

\maketitle

\begin{enumerate}[1)]
    \item Show that if the conjugacy classes of a finite group \(G\) have size at most 4, then \(G\) is solvable. \\
    
    \textbf{Solution}: (Thanks to \href{https://math.stackexchange.com/users/2820/derek-holt}{Derek Holt} on stackexchange for the hint which enabled me to answer this - see \href{https://math.stackexchange.com/a/4208528/870855}{here} for a solution which drops the finiteness hypothesis). \\
    
    We will use the following fact in this proof: Let \(G\) be a group with normal subgroup \(N\). Then:
    \[
    G \text{ is solvable} \iff N \text{ is solvable and } G/N \text{ is solvable}
    \]
    Let \(G\) be finite and have conjugacy classes of size at most 4. Then this means that:
    \[
    [ G \colon C_{G}(g) ] \leq 4 \quad \forall g \in G
    \]
    If all conjugacy classes are of size \(1\), then \(G\) is abelian, so then \( [G,G] = \set{e}\) so \(G\) is solvable. If not, take some \(g \in G\) with conjugacy class of size \(n = 2,3 \text{ or } 4\). Then \(G\) acts transitively on the set of cosets of \(C_{G}(g)\), of which there are \(n_{1}\) of. So we get a non-trivial homomorphism \( \sigma_{1} \colon G \to S_{n} \), whence:
    \[
    \faktor{G}{\ker \sigma_{1}} \cong \im \sigma_{1} \leq S_{n_1}
    \]
    So \(\faktor{G}{\ker \sigma_{1}}\) is isomorphic to a subgroup of \(S_{4}\) (as this contains \(S_3\) and \(S_2\) as subgroups), which is non-trivial due to the transitivity of the action. So \(\ker \sigma_{1}\) is a subgroup of \(G\) not equal to \(G\). As a subgroup of \(G\), its conjugacy classes are also of size at most 4, because conjugacy classes cannot get bigger within subgroups. We claim that \(\ker \sigma_{1}\) is solvable: if it's abelian, we're done, if not we take some \(g \in \ker \sigma_{1}\) with non-trivial conjugacy class, of size \(n_{2} \in \set{2,3,4}\). Then we get a transitive action from \(ker \sigma_{1}\) to \(S_{n_2}\) so then:
    \[
    \faktor{\ker \sigma_{1}}{\ker \sigma_{2}} \cong \im \sigma_{2} \leq S_{n_2}
    \]
    So the quotient \( \ker \sigma_{1}/ \ker \sigma_{2}\) is solvable, and \(\ker \sigma_{2} \lhd \ker \sigma_{1}\) again has conjugacy classes of all of size at most 4. So we keep iterating to get:
    \[
     G \triangleright \ker \sigma_{1}  \triangleright \dots \triangleright \ker \sigma_{m-1} \triangleright \ker \sigma_{m}
    \]
    As these inclusions are proper and \(G\) is finite, this set of inclusions does indeed terminate eventually because the order decreases each time: so \(\ker \sigma_{m}\) is abelian for some \(m \in \NN\) (which precisely means that all of its conjugacy classes are singletons). Then by induction \(\ker \sigma_{m-1}\) is solvable as \(\ker \sigma_{m}\) and \(\ker \sigma_{m-1} / \ker \sigma_{m}\) is solvable (subgroup of \(S_{n_{m}}\) where \( n_{m} \leq 4\)). So we have \(\ker \sigma_{m} \text{ solvable } \implies \ker \sigma_{m-1} \text{ solvable } \implies \dots \implies \ker \sigma_{1} \text{ solvable } \implies G \text{ solvable }\).
    
    \item Show that if \(F\) is a nontrivial free group, then \(F\) has a proper subgroup of finite index.\\
    
    \textbf{Solution}: \\
    Let \(A \neq \emptyset\) be a set, with \(F = F(A)\). Then consider:
    \[
    f \colon F(A) \twoheadrightarrow F(A)^{\textbf{Ab}} =:G
    \]
    A surjective homomorphism from \(F(A)\) to \(G\), the free \textit{Abelian} group generated by \(A\): Note that this exists by the universal property of free groups: We have the injective set function \(\iota \colon A \to F(A)^{\textbf{Ab}}\), so thus \(f\) is the (unique) surjective group homomorphism making the following diagram commute:
    \[
    \begin{tikzcd}
    F(A) \arrow[rr,"f"]  & &F(A)^{\textbf{Ab}} \\
        & & \\
        A \arrow[hookrightarrow,uu] \arrow[hookrightarrow,uurr,"\iota"] & &
    \end{tikzcd}
    \]
    where the unmarked arrow is the obvious inclusion \(A \to F(A)\). Now, \(G\) consists of elements of the form:
    \[
    g = \sum_{a \in A} \lambda_{a} a
    \]
    Where \(a \in A\) and \(\lambda_{a} \in \ZZ\), non-zero for only finitely many \(a \in A\). Pick some arbitrary \(b \in A\). Then consider the subgroup \(H\) of \(G\) generated by:
    \[
    \set{2 b} \cup \bigcup_{a \in A\setminus \set{b}} a
    \]
    (note this subgroup is automatically normal because \(G\) is Abelian). Then considering cosets of \(H\) in \(G\) we have that in \(G/H\), taking some arbitrary \(g \in G\)
    \[
    g = \sum_{a \in A} \lambda_{a} a = \lambda_{b} b + \sum_{a \in A \setminus \set{b}} a \equiv \lambda_{b} \equiv 
    \begin{cases}
    0 \text{ if \(\lambda_{b}\) is even}\\
    b \text{ if \(\lambda_{b}\) is odd}
    \end{cases}
    \]
    Then \(H\) obviously has only \(2\) cosets. So we have a surjection \(\pi \colon G \to \ZZ_{2}\). But then \(\pi \circ f\) is a surjection from \(F(A)\) to \(\ZZ_{2}\), so by the first isomorphism theorem:
    \[
    \faktor{F(A)}{\ker(\pi \circ f)} \cong \ZZ_{2}
    \]
    i.e. \(\ker(\pi \circ f)\) is an index 2 subgroup of \(F(A)\), so \(F(A)\) has a proper subgroup of finite index (proper as the above isomorphism isn't with the trivial group, so \(\ker(\pi \circ f)\) can't be all of \(F(A)\). As \(A\) was an arbitrary non-singleton set (only non-singletons generate non-trivial free groups) we have the result.
    \\
    
    \item Show that if \(R\) is PID and \(S\) is an integral domain containing no subfield, then any homomorphism \(\varphi \colon R \to S\) is injective. \\
    
    \textbf{Solution}: \\
    By the first isomorphism theorem for rings:
    \[
    \faktor{R}{\ker(\varphi)} \cong \im \varphi
    \]
    Note that \(\im \varphi\) is a subring of \(S\) (and so is also an integral domain). Then we combine the facts that for any commutative ring \(R\) with non-trivial ideal \(I\):
    \[
    \faktor{R}{I} \text{ is a field} \iff \text{ \(I\) is a maximal ideal of \(R\)}
    \]
    And that PID \(\implies\) commutative ring, and that in a PID an ideal \(I\) is maximal iff it is prime. We also have for arbitrary commutative ring \(R\): 
    \[
    \faktor{R}{I} \text{ is an integral domain} \iff \text{ \(I\) is a prime ideal of \(R\)}
    \]
    For any proper ideal \(I\). So \(S\) being an integral domain implies that \(\im{\varphi}\) is an integral domain (as it's a subring of \(S\)), which implies that \(\ker(\varphi)\) is a prime ideal, which in turn implies that it's maximal as \(R\) is a PID, which then means that \(\im(\varphi) \cong R/\ker(\varphi)\) is a field. So we must have that \(\ker(\varphi)\) is \textit{not} a proper ideal of \(R\): if it's \(R\) then this is impossible as this implies that \(1_{R}\) is in \(\ker(\varphi)\), but ring homomorphisms send \(1\) to \(1\), so \(S\) is the zero ring, which is \textit{not} an integral domain. So the only other option is that \(\ker(\varphi) = \set{0}\) i.e. \(\varphi\) is injective.
    
    \item Let \(A\) be an \(n \times n\) matrix over a field \(K\). Show that if \(A\) has exactly one invariant factor, then any matrix \(B\) that commutes with \(A\) must be a polynomial in \(A\) (That is, show that if \(BA = AB\), then \(B = p(A)\) for some \(p(x) \in K[x]\).)\\
    
    \textbf{Solution}: \\
    
    \item Suppose that \(f \in \ZZ[x]\) is a monic irreducible polynomial of degree 4. Suppose further that there is a complex number \(\alpha\) such that both \(\alpha\) and \(\alpha^2\) are roots of \(f\). What is \(f\)?\\
    
    \textbf{Solution}: \\
    Consider \(\alpha = \zeta_5\), primitive 5th root of unity in \( \mathbb{C}\). Then:
    \[
    x^5 - 1 = (x - \zeta_5)(x-\zeta_5^2)(x-\zeta_5^3)(x-\zeta_5^4)(x-1)
    \]
    so 
    \[
    f(x) = (x - \zeta_5)(x-\zeta_5^2)(x-\zeta_5^3)(x-\zeta_5^4) = \frac{x^5-1}{x-1} = x^4 + x^3 + x^2 +x +1 \in \ZZ[x]
    \]
    Has \(\alpha\) and \(\alpha^2\) as roots and is of degree 4. It is irreducible as:
    \[
    f(x+1) = \frac{(x+1)^5 - 1}{x+1-1}
    \]
    is irreducible by Eisenstein's criterion with \(p=5\). So \(f\) satisfies the hypotheses of the question.  \\
    
    
    \item Let \(\zeta\) be a primitive 8th root of unity and let \(K = \QQ[\zeta]\). Determine \(\Gal(K/\QQ)\) and all the intermediate fields between \(\QQ\) and K.  \\
    
    \textbf{Solution}: \\
    
    
    
\end{enumerate}
\end{document}