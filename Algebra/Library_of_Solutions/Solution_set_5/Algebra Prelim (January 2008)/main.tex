\documentclass[11pt]{article}
 
\usepackage[margin=2cm]{geometry} 
\usepackage[oldstylenums]{kpfonts}
\usepackage{mathrsfs} %includes mathscript font
\usepackage{amsmath,amsthm,amssymb}
\usepackage[utf8]{inputenc}
\usepackage[english]{babel}
\usepackage{tikz}
\usepackage{mathtools}
\usetikzlibrary{matrix}
\usepackage{graphicx}
\usepackage{subcaption}
\usepackage{float}
\usepackage{stmaryrd}
\usepackage{hyperref}
\hypersetup{
    colorlinks=true,
    linkcolor=blue,
    filecolor=magenta,      
    urlcolor=cyan,
}
\usepackage{titling}

%setting legth for author/date
\setlength{\droptitle}{-5em}

%mapsto arrow with labeling on top
\makeatletter
\newcommand{\xMapsto}[2][]{\ext@arrow 0599{\Mapstofill@}{#1}{#2}}
\def\Mapstofill@{\arrowfill@{\Mapstochar\Relbar}\Relbar\Rightarrow}
\makeatother

%floor function
\newcommand{\floor}[1]{\lfloor #1 \rfloor}

%standard sets
\newcommand{\N}{\mathbb{N}}
\newcommand{\Z}{\mathbb{Z}}
\newcommand{\R}{\mathbb{R}}
\newcommand{\C}{\mathbb{C}}
\newcommand{\Hq}{\mathbb{H}}
\newcommand{\F}{\mathbb{F}}
\newcommand{\Q}{\mathbb{Q}}

%Quaternions
\newcommand{\1}{\textbf{1}}
\newcommand{\bi}{\textbf{i}}
\newcommand{\bj}{\textbf{j}}
\newcommand{\bk}{\textbf{k}}

 
\usepackage{amsthm}

\newtheorem{stheorem}{Theorem}[section]
\newtheorem{scor}{Corollary}[section]
\newtheorem{slemma}{Lemma}[section]
\newtheorem{sprop}{Proposition}[section]
\newtheorem{sconj}{Conjecture}[section]
\newtheorem{sex}{Exercise}[section]
\newtheorem{sdefi}{Definition}[section]
\newtheorem{sexa}{Example}[section]
\newtheorem{sprob}{Problem}[section]


\newtheorem{theorem}{Theorem}
\newtheorem{cor}{Corollary}
\newtheorem{lemma}{Lemma}
\newtheorem{prop}{Proposition}
\newtheorem{conj}{Conjecture}
\newtheorem{ex}{Exercise}
\newtheorem{defi}{Definition}
\newtheorem{exa}{Example}
\newtheorem{prob}{Problem}


\newtheorem*{theorem*}{Theorem}
\newtheorem*{cor*}{Corollary}
\newtheorem*{lemma*}{Lemma}
\newtheorem*{prop*}{Proposition}
\newtheorem*{conj*}{Conjecture}
\newtheorem*{ex*}{Exercise}
\newtheorem*{defi*}{Definition}
\newtheorem*{exa*}{Example}

\newtheorem*{sol*}{\textit{Solution}}

%transverse sign as in Guillemin & Pollack
\newcommand{\transv}{\mathrel{\text{\tpitchfork}}}
\makeatletter
\newcommand{\tpitchfork}{%
  \vbox{
    \baselineskip\z@skip
    \lineskip-.52ex
    \lineskiplimit\maxdimen
    \m@th
    \ialign{##\crcr\hidewidth\smash{$-$}\hidewidth\crcr$\pitchfork$\crcr}
  }%
}
\makeatother



\begin{document}

\title{\LARGE \textbf{CU Boulder: \textit{Algebra} %subject of exam
Prelim \\ \textit{January 2008}} %date exam was administered
\vspace{-.75cm}}% vspace sets margin between title and author
\author{Juan Moreno
} 
\date{\vspace{-0.45cm}April 2019} % can input date if desired, vspace sets margin between author and date

 
\maketitle

%remove abstract title
\renewcommand{\abstractname}{\vspace{-\baselineskip}}
%insert description/ abstract
\begin{abstract}
\noindent These are my solutions to the questions on the CU Boulder \textit{Algebra} preliminary exam from \textit{January 2008} found  \href{http://math.colorado.edu/documents/graduate/prelim/Algebra_Jan_2008.pdf}{here}. I worked on these solutions over the summer of 2019 in preparation for the preliminary exam in the Fall 2019. Please send any questions, comments, or corrections to \href{mailto: juan.moreno-1@boulder.edu}{juan.moreno-1@boulder.edu.} \\
\end{abstract}


\begin{prob}
Let G be a nonabelian finite simple group, and let p be a prime divisor of its order $|G|$. Show that if the number of Sylow $p$-subgroups of $G$ is $n$, then $|G|$ divides $n!$.

\begin{proof}
If $p$ is a prime divisor of $|G|$ then $\textnormal{Syl}_p(G) \neq \emptyset$ and $G$ acts on this set of Sylow $p$-subgroups by conjugation. This gives rise to a homomorphism $G\rightarrow S_n$, where $n = |\textnormal{Syl}_p(G)|$. Since $G$ is simple and the kernel of a homomorphism is a normal subgroup, either this homomorphism is injective, in which case $G$ can be viewed as a subgroup of $S_n$ and the result follows from Lagrange's Theorem, otherwise the kernel of this homomorphism is all of $G$. We show that the latter case cannot be.

In this latter case, we have for all $g\in G$, $gPg^{-1} = P, \forall P\in\textnormal{Syl}_p(G)$, implying every Sylow $p$-subgroup is normal in $G$. Since we have already established the set of such subgroups is nonempty and $1$ is not a prime, we must have that $|G| = p^\alpha$ for some positive integer $\alpha$. The class equation for $G$ then reads \[|G| = p^\alpha = |Z(G)| + \sum_{\mathcal{O}\in\mathcal{C}}|\mathcal{O}|,\] where $Z(G)$ is the center of $G$ and $\mathcal{C}$ is the set of conjugacy classes of order $>1$. Since $p| p^\alpha$, we must have that $p$ divides the right side of the class equation. By the Orbit-Stabilizer Theorem, the order of the orbits $\mathcal{O}$ must divide $|G| = p^\alpha $ and since these orders are greater that $1$, we have that $p$ divides the sum on the right side of the class equation. It follows that $p$ must also divide $|Z(G)|$ so that the center of $G$ is a nontrivial subgroup. Since the center of a group is always normal, if we are to reconcile this with the fact that $G$ is simple, we must have that $Z(G) = G$, implying $G$ must be abelian.
\end{proof}
\end{prob}

\begin{prob}
Let $G$ be a finite solvable group. Show that

\noindent (a) $G$ has a nontrivial abelian normal subgroup of prime power order.
\begin{proof}
Let $H$ be a minimal nontrivial normal subgroup of $G$. Then $H$ must also be solvable, so its derived series must eventually trivialize. Note that $H' = [H,H]\leq H$ is a characteristic subgroup of $H$, and since $H\trianglelefteq G$, we have that $H'\trianglelefteq G$. Thus $H' = 1$, implying $H$ is abelian. Now consider, for any prime $p$ dividing $|H|$, $H_p = \langle x\in H| x^p = 1\rangle$. This is a characteristic subgroup of $H$ since any automorphism preserves order. Further, by Cauchy's theorem, this subgroup is nontrivial. Thus, by minimality of $H$, $H_p = H$. It follows that $H$ is a nontrivial abelian normal subgroup of $G$ of prime power order. 
\end{proof}

\noindent (b) every maximal proper subgroup of $G$ has prime power index in $G$

\begin{proof}
Note that the result holds for the trivial group $G = 1$ and the only group of order $2$, $Z_2$. Proceeding by induction on the order of $G$, let $H\leq G$ be maximal. Suppose first that $H$ contains a minimal nontrivial abelian normal subgroup of prime power order as in part (a), $N$. Then $H/N\leq G/N$ is maximal (lattice isomorphism) and $G/N$ is a solvable group of order strictly less than $|G|$ so that the induction hypthesis implies the index of $H/N$ in $G/N$ is a prime power. It follows that the index of $H$ in $G$ is a prime power. Now suppose $H$ does not contain any such minimal subgroup. Then for any such minimal subgroup, $N$, $NH$ is a subgroup of $G$ containing $H$ so that by maximality of $H$ and the fact that $H$ does not contain $N$, $NH = G$. THus \[|NH| = \frac{|N||H|}{|N\bigcap H|} = |G| \] \[\implies \frac{|G|}{|H|} = \frac{|N|}{|N\bigcap H|},\] and $\frac{|N|}{|N\bigcap H|}$ is a prime power. 
\end{proof}
\end{prob}

\begin{prob}
Let $R$ be a UFD such that any ideal generated by two elements of $R$ is principal.
Prove that $R$ is a PID.

\begin{proof}
Let $I$ be any ideal of $R$ and let $a\in I$ be an element with a minimal number of irreducible factors. Such an element always exists since in a UFD every element can be expressed as a finite product of irreducibles unique up to multiplication by a unit and the number of such irreducible factors is unique. If $b\in I\setminus (a)$, then $(a,b) = (d)\subset I$, where $d$ is a greatest common divisor of $a$ and $b$. However this contradicts the minimality of the number of irreducible factors of $a$ so that in fact any $b\in I$ is contained in $(a)$. Thus $I = (a)$.  
\end{proof}
\end{prob}

\begin{prob}
Let $A$ be an $n\times n $ matrix over $\C$ such that $\textnormal{Tr}(A^k) = 0$ for all $k>0$. Show that $A^n = 0$. 

\begin{sol*}
\textnormal{Let $c_A(x) = x^n + a_{n-1}x^{n-1} + ... + a_1 x + a_0$ be the characteristic polynomial of $A$. Then $c_A(A) = 0$ implying \[\textnormal{Tr}(c_A(A)) = \textnormal{Tr}(A^n + a_{n-1}A^{n-1} + ... + a_1 A + a_0)\] \[= \textnormal{Tr}(A^n) + a_{n-1}\textnormal{Tr}(A^{n-1}) + ... + a_1\textnormal{Tr}(A) + a_0\textnormal{Tr}(I)\] \[= a_0\cdot n = 0 \]\[\implies a_0 = 0.\] Thus $c_A(x) = xc_1(x) = x(x^{n-1} + a_{n-1}x^{n-2} + ... + a_2 x + a_1)$, implying either $A = 0$ or $A$ satisfies $c_1(A) = 0$. Proceeding as before, we get that $a_0 = a_1 = ... = a_{n-1} = 0$ so that $c_A(x) = x^n$, implying $A^n = 0$.  
}
\end{sol*}
\end{prob}


\begin{prob}
Find the splitting field of $x^4 + x^3 + 1$ over the $32$-element field. 

\begin{sol*}
\textnormal{ First note that since this polynomial lies in $\mathbb{F}_2[x]\subset\mathbb{F}_{32}[x]$, it suffices to find the splitting field of this polynomial over $\mathbb{F}_2$, say $K$, and then compute the composite $K\mathbb{F}_{32}$. Now our polynomial $f(x) = x^4 + x^3 +1$ is irreducible over $\mathbb{F}_2$ since it has no roots in this field and the only possible irreducible factor is $x^2 + x + 1$ which does not square to $f$. Thus $\mathbb{F}_2[x]/(f)\cong\mathbb{F}_{2^4}$ is the splitting field of $f$ over $\mathbb{F}_2$. Then $\mathbb{F}_{2^4}\mathbb{F}_{2^5} = \mathbb{F}_{2^{20}}$ is the splitting field of $f$ over $\mathbb{F}_{32}$. 
}
\end{sol*}
\end{prob}


\begin{prob}
True of false? Justify your answer.

\noindent (i) Every field extension of degree $2$ is Galois.

\noindent\underline{Claim:} False

\begin{proof}
If char$F\neq 2$ then any degree $2$ extension of $F$ is of the form $F(\sqrt{D})$ for some $D\in F$. This is the splitting field of the irreducible polynomial $x^2 - D\in F[x]$ hence is a Galois extension. However, if char$F = 2$ we have the following counterexample. Consider $x^2 - t\in \mathbb{F}_2(t)[x]$. Since $(x + \sqrt{t})^2 = x^2 -t$, this polynomial is not separable and so it's degree $2$ splitting field is not Galois.
\end{proof}

\noindent (ii) Every algebraically closed field is infinite.

\noindent\underline{Claim:} True
\begin{proof}

\noindent (iii) If $K$ is a finite field then it is a finite extension of its prime subfield $F$. The prime subfield of $K$ must be $\mathbb{F}_p$ for some prime $p$ otherwise, char$F = 0$ and the prime subfield will be infinite, contradicting that $K$ is finite. Thus, $K \cong \mathbb{F}_{p^n}$ for some $n$. This field is not algebraically closed since, for example $\mathbb{F}_{p^n}\subset\mathbb{F}_{p^{2n}}$ is a degree $2$ Galois extension so that $\mathbb{F}_{p^{2n}}$ is the splitting field of some irreducible polynomial in $\mathbb{F}_{p^n}[x]$. 

\end{proof}


\noindent (iii) If $\alpha = \sqrt[^5]{2 + i} + \sqrt[^5]{2-i}$, then Gal$(\Q [\alpha]/\Q )\cong S_5$. 

\noindent\underline{Claim:} False
\begin{proof}
Consider the following diagram of field extensions. Since $Q(\sqrt[^5]{2 + i}, \sqrt[^5]{2-i})$ is the composite of the left and right fields in the diagram which are each of degree $5$ over $\Q(i)$, we have that $[\Q(\sqrt[^5]{2 + i}, \sqrt[^5]{2-i}):\Q(i)]$ is at most $25$. It follows that $[\Q(\sqrt[^5]{2 + i}, \sqrt[^5]{2-i}):\Q]$ is at most $50$. Thus $[\Q(\alpha):\Q]$ is at most $50$ so that the order of the Galois group of $\Q(\alpha)/\Q$ is strictly less than $|S_5|$. 

\begin{center}
\begin{tikzpicture}[node distance = 2cm, auto]

\node (Q) {$\Q$};

\node (Qi) [above of=Q] {$\Q(i)$};

\node (Qalpha) [above of=Qi] {$\Q(\alpha)$};

\node (Q1) [left of=Qalpha] {$\Q(\sqrt[^5]{2+i})$};

\node (Q2) [right of=Qalpha] {$\Q(\sqrt[^5]{2-i})$};

\node (Qtop) [above of=Qalpha] {$\Q(\sqrt[^5]{2+5},\sqrt[^5]{2-i})$};

\draw[-] (Q) to node {$2$} (Qi);
\draw[-] (Qi) to node {} (Qalpha);
\draw[-] (Qi) to node {$5$} (Q1);
\draw[-] (Qi) to node {$5$} (Q2);
\draw[-] (Q1) to node {} (Qtop);
\draw[-] (Q2) to node {} (Qtop);
\draw[-] (Qalpha) to node {} (Qtop);

\end{tikzpicture}
\end{center}
\end{proof}

\end{prob}

\end{document}
