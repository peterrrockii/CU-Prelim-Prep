\documentclass[11pt]{article}
 
\usepackage[margin=2cm]{geometry} 
\usepackage[oldstylenums]{kpfonts}
\usepackage{mathrsfs} %includes mathscript font
\usepackage{amsmath,amsthm,amssymb}
\usepackage[utf8]{inputenc}
\usepackage[english]{babel}
\usepackage{tikz}
\usepackage{mathtools}
\usetikzlibrary{matrix}
\usepackage{graphicx}
\usepackage{subcaption}
\usepackage{float}
\usepackage{stmaryrd}
\usepackage{hyperref}
\hypersetup{
    colorlinks=true,
    linkcolor=blue,
    filecolor=magenta,      
    urlcolor=cyan,
}
\usepackage{titling}

%setting legth for author/date
\setlength{\droptitle}{-5em}

%mapsto arrow with labeling on top
\makeatletter
\newcommand{\xMapsto}[2][]{\ext@arrow 0599{\Mapstofill@}{#1}{#2}}
\def\Mapstofill@{\arrowfill@{\Mapstochar\Relbar}\Relbar\Rightarrow}
\makeatother

%floor function
\newcommand{\floor}[1]{\lfloor #1 \rfloor}

%standard sets
\newcommand{\N}{\mathbb{N}}
\newcommand{\Z}{\mathbb{Z}}
\newcommand{\R}{\mathbb{R}}
\newcommand{\C}{\mathbb{C}}
\newcommand{\Hq}{\mathbb{H}}
\newcommand{\F}{\mathbb{F}}

%Quaternions
\newcommand{\1}{\textbf{1}}
\newcommand{\bi}{\textbf{i}}
\newcommand{\bj}{\textbf{j}}
\newcommand{\bk}{\textbf{k}}

 
\usepackage{amsthm}

\newtheorem{stheorem}{Theorem}[section]
\newtheorem{scor}{Corollary}[section]
\newtheorem{slemma}{Lemma}[section]
\newtheorem{sprop}{Proposition}[section]
\newtheorem{sconj}{Conjecture}[section]
\newtheorem{sex}{Exercise}[section]
\newtheorem{sdefi}{Definition}[section]
\newtheorem{sexa}{Example}[section]
\newtheorem{sprob}{Problem}[section]


\newtheorem{theorem}{Theorem}
\newtheorem{cor}{Corollary}
\newtheorem{lemma}{Lemma}
\newtheorem{prop}{Proposition}
\newtheorem{conj}{Conjecture}
\newtheorem{ex}{Exercise}
\newtheorem{defi}{Definition}
\newtheorem{exa}{Example}
\newtheorem{prob}{Problem}


\newtheorem*{theorem*}{Theorem}
\newtheorem*{cor*}{Corollary}
\newtheorem*{lemma*}{Lemma}
\newtheorem*{prop*}{Proposition}
\newtheorem*{conj*}{Conjecture}
\newtheorem*{ex*}{Exercise}
\newtheorem*{defi*}{Definition}
\newtheorem*{exa*}{Example}

\newtheorem*{sol*}{\textit{Solution}}

%transverse sign as in Guillemin & Pollack
\newcommand{\transv}{\mathrel{\text{\tpitchfork}}}
\makeatletter
\newcommand{\tpitchfork}{%
  \vbox{
    \baselineskip\z@skip
    \lineskip-.52ex
    \lineskiplimit\maxdimen
    \m@th
    \ialign{##\crcr\hidewidth\smash{$-$}\hidewidth\crcr$\pitchfork$\crcr}
  }%
}
\makeatother



\begin{document}

\title{\LARGE \textbf{CU Boulder: \textit{Algebra} %subject of exam
Prelim \\ \textit{August 2008}} %date exam was administered
\vspace{-.75cm}}% vspace sets margin between title and author
\author{Juan Moreno
} 
\date{\vspace{-0.45cm}April 2019} % can input date if desired, vspace sets margin between author and date

 
\maketitle

%remove abstract title
\renewcommand{\abstractname}{\vspace{-\baselineskip}}
%insert description/ abstract
\begin{abstract}
\noindent These are my solutions to the questions on the CU Boulder \textit{Algebra} preliminary exam from \textit{August 2008} found  \href{http://math.colorado.edu/documents/graduate/prelim/Algebra_Aug_2008.pdf}{here}. I worked on these solutions over the summer of 2019 in preparation for the preliminary exam in the Fall 2019. Please send any questions, comments, or corrections to \href{mailto: juan.moreno-1@boulder.edu}{juan.moreno-1@boulder.edu.} \\
\end{abstract}

\begin{prob}
(17 pts) Show that a group G of order $2^3 \cdot 5 \cdot 13$ cannot be simple.

\begin{proof}
By Sylow's Theorem, the number of Sylow $13$-subgroups of $G$ must satisfy $n_{13}\equiv 1(\textnormal{mod} 13) $ and $n_{13}\big| 2^3\cdot 5$. The only possibilities are $n_{13} = 1$ or $40$. If $n_{13} = 40$, then since the order of these Sylow $13$-subgroups is prime, these subgroups must all have trivial intersection. So we count $12\cdot 40 = 480$ distinct elements of order $13$ in $G$. Similarly the number of Sylow $5$-subgroups of $G$ must satisfy $n_5 \equiv 1(\textnormal{mod}5)$ and $n_5\big| 104$. The only possibilities are $n_5 = 1$ or $26$. If $n_5 = 26$ then by the same reasoning as above we count $4\cdot 26 = 104$ distinct elements of order $5$. If both $n_{13} = 40$ and $n_5 = 26$ then we would have $480 + 104 = 584 > |G|$ distinct elements, a contradiction. We then have that at least one of these must be $1$, implying $G$ has a unique Sylow subgoup which, by Sylow's Theorem must be normal. Thus $G$ cannot be simple.
\end{proof}
\end{prob}

\begin{prob}
(17 pts) Let $G$ be a finite group which acts on a set $S$ on both the left and the right. For an element $s \in S$, let $Gs$ and $sG$ denote the orbit of $s$ under these respective actions.These actions can be combined into a single (left) action of$G \times G$ on $S$ via $(g, h)s = gsh^{-1}$. The corresponding orbit of $s$ under this action is denoted $GsG$. There are two independent questions one wants to answer about such orbits: what is their size, and how many of them are there?

\noindent (a) (12 pts) Show that for $s\in S$ the size of $GsG$ is \[ | GsG| = \frac{|Gs||sG|}{|Gs\bigcap sG|}.\]

\begin{proof}

\end{proof}
\end{prob}


\end{document}
