\section{Measures}
\setcounter{thm}{0}

\dfn Let $X$ be a nonempty set. An \textbf{\textit{algebra}} of sets on $X$ is a nonempty collection $\A$ of subsets of $X$ that is closed under finite unions and compliments. In other words, if $E_1,\ldots,E_n\in \A$, then $\bigcup_1^n E_j\in \A$; and if $E\in \A$, then $E^c\in\A$. A $\boldsymbol\sig$\textbf{\textit{-algebra}} is an algebra that is closed under countable unions. 

\dfn If $X$ us any set, $\PP(X)$ and $\{\es, X\}$ are $\sig$-algebras. If $X$ is uncountable, then
\[\A = \{E\subset X: E\text{ is countable or $E^c$ is countable}\}\]
is a $\sig$-algebra, called the \textbf{\textit{$\boldsymbol\sig$-algebra of countable or co-countable sets.}}

\dfn If $\EE$ is any subset of of $\PP(X)$, there is a unique smallest $\sig$-algebra $\MM(\EE)$ containing $\EE$, namely, the intersection of all $\sig$-algebras containing $\EE$. $\MM(\EE)$ is called the $\sig$-algebra \textbf{\textit{generated}} by $\EE$.\\


\begin{lem}
If $\EE\subset\MM(\FF)$ then $\MM(\EE)\subset \MM(\FF)$.
\end{lem}

\vs

\dfn Let $X$ be a metric space. The $\sig$-algebra generated by the family of open sets in $X$ is called the \textbf{\textit{Borel $\boldsymbol\sig$-algebra}} on $X$ and is denoted by $\BB_X$. Its members are called \textbf{\textit{Borel sets.}}

\textbf{Notation.} There is a standard terminology for the hierarchy of Borel sets. A countable intersection of open sets is called a $\boldsymbol{G_\de}$ set; a countable union of closed sets is called a $\boldsymbol{F_\sig}$ set; a countable union of $G_\de$ sets is called a $\boldsymbol{G_{\de\sig}}$ set; a countable intersection of $F_\sig$ sets is called a $\boldsymbol{F_{\sigma\de}}$ set; and so forth with $\de$ corresponding to countable intersections and $\sig$ corresponding to countable unions.\\

\begin{prop}
$\BB_\R$ is generated by each of the following:
\begin{enumerate}
\item the open intervals: $\EE_1 = \{(a,b)\ |\ a < b\}$,
\item the closed intervals: $\EE_2 = \{[a,b]\ |\ a < b\}$,
\item the half-open intervals: $\EE_3 = \{[a,b)\ |\ a < b\}$ or $\EE_4 = \{(a,b]\ |\ a < b\}$,
\item the open rays: $\EE_5 = \{(a,\infty)\ |\ a \in \R\}$ or $\EE_6 = \{(-\infty,a)\ |\ a \in \R\}$,
\item the closed rays: $\EE_7 = \{[a,\infty)\ |\ a \in \R\}$ or $\EE_8 = \{(-\infty,a]\ |\ a \in \R\}$.
\end{enumerate}
\end{prop}

\vs

\dfn Let $\{X_\al\}_{\al\in A}$ be an indexed collection of nonempty sets, $X = \prod_{\al\in A} X_\al$, and $\pi_\al:X\ra X_\al$ the coordinate maps. If $\MM_\al$ is a $\sig$-algebra on $X_\al$ for each $\al$, the \textbf{\textit{product $\boldsymbol\sig$-algebra}} on $X$ is the $\sig$-algebra generated by 
\[\{\pi_\al\inv(E_\al)\ |\ E_\al \in \MM_\al,\ \al\in A\}.\]
We denote this $\sig$-algebra by $\bigotimes_{\al\in A}\MM_\al$\\

\begin{prop}
If $A$ is countable, then $\bigotimes_{\al\in A}\MM_\al$ is the $\sigma$-algebra generated by $\left\{\prod_{\al\in A} E_\al \ |\ E_\al \in \MM_\al \right\}$.
\end{prop}

\vs

\begin{prop}
Suppose that $\MM_\al$ is generated by $\EE_\al,\ \al\in A$. Then $\bigotimes_{\al\in A}\MM_\al$ is generated by $\FF_1 = \{\pi_\al\inv(E_\al)\ |\ E_\al \in \EE_\al,\ \al \in A\}$. If $A$ is countable and $X_\al\in \EE_\al$ for all $\al,\ \bigotimes_{\al\in A}\MM_\al$ is generated by $\FF_2 = \left\{\prod_{\al\in A} E_\al\ |\ E_\al \in \EE_\al\right\}$.
\end{prop}

\vs 

\begin{prop}
Let $X_1,\ldots, X_n$ be metric spaces and let $\prod_1^n X_j$, equipped with the product metric. Then $\bigotimes_1^n \BB_{X_j} \subset B_X$. If the $X_j$'s are separable, then $\bigotimes_1^n \BB_{X_j} = \BB_X$.
\end{prop}

\vs 

\begin{cor}
$\BB_{\R^n} = \bigotimes_1^n \BB_\R$.
\end{cor}

\vs 
\dfn An \textbf{\textit{elementary family}} is a collection $\EE$ of subsets of $X$ such that
\begin{itemize}
\item $\es\in \EE$,
\item if $E,F\in \EE$ then $E\cap F\in \EE$,
\item if $E\in \EE$ then $E^c$ is a finite disjoint union of members of $\EE$.
\end{itemize}

\vs 

\begin{prop}
If $\EE$ is an elementary family, the collection $\A$ of finite disjoint unions of members of $\EE$ is an algebra.
\end{prop}

\vs 
\dfn Let $X$ be a set equipped with a $\sig$-algebra $\MM$. A \textbf{\textit{measure}} on $\MM$ is a function $\mu:\MM\ra [0,\infty]$ such that 
\begin{enumerate}
\item[i.] $\mu(\es) = 0$,
\item[ii.] if $\{E_j\}_1^\infty$ is a sequence of disjoint sets in $\MM$, then $\mu(\bigsqcup_1^\infty E_j) = \sum_1^\infty \mu(E_j)$.
\end{enumerate}
Property (ii) is called \textbf{\textit{countable additivity}}. It implies \textbf{\textit{finite additivity}}:
\begin{enumerate}
\item[ii$\p$.] if $E_1,\ldots,E_n$ are disjoint sets in $\MM$, then $\mu(\bigsqcup_1^nE_j) = \sum_1^n\mu(E_j)$.
\end{enumerate}
A function $\mu$ that satisfies (i) and (ii$\p$) but not necessarily (ii) is called a \textbf{\textit{finitely additive measure}}

\dfn If $X$ is a set and $\MM\subset \PP(X)$ is a $\sig$-algebra, $(X,\MM)$ is called a \textbf{\textit{measurable space}} and the sets in $\MM$ are called \textbf{\textit{measurable sets}}. If $\mu$ is a measure on $(X,\MM)$, then $(X, \MM, \mu)$ is called a \textbf{\textit{measure space}}.


\dfn Let $(X, \MM, \mu)$ be a measure space. If $\mu(X) < \infty$, then $\mu$ is called \textbf{\textit{finite}}. If $X = \bigcup_1^\infty E_j$ where $E_j\in \MM$ and $\mu(E_j) < \infty$ for all $j$, $\mu$ is called \textbf{\textit{$\boldsymbol\sig$-finite}}. \hl{If for each $E\in \MM$ with $\mu(E) = \infty$ there exists $F\in \MM$ with $F\subset E$ and $0 < \mu(F) < \infty$, $\mu$ is called \textbf{\textit{semifinite}}.}\\

\begin{thm}
Let $(X, \MM, \mu)$ be a measure space
\begin{enumerate}
\item[a.] \textbf{(Monotonicity)} If $E, F \in \MM$ and $E\subset F$, then $\mu(E)\leq \mu(F)$.
\item[b.] \textbf{(Subadditivity)} If $\{E_j\}_1^\infty \subset \MM$, then $\mu(\bigcup_1^\infty E_j) \leq \sum_1^\infty \mu(E_j)$.
\item[c.]  \textbf{(Continuity from below)} If $\{E_j\}_1^\infty \subset \MM$ and $E_1\subset E_2\subset \cdots$, then $\mu(\bigcup_1^\infty E_j) = \lim_{j\ra \infty}\mu(E_j)$.
\item[d.]  \textbf{(Continuity from above)} If $\{E_j\}_1^\infty \subset \MM$ and $E_1\supset E_2\supset \cdots$, and $\mu(E_1) < \infty$, then $\mu(\bigcap_1^\infty E_j) = \lim_{j\ra \infty}\mu(E_j)$.
\end{enumerate}
\end{thm}

\vs 

\dfn If $(X, \MM, \mu)$ is a measure space, a set $E\in \MM$ such that $\mu(E) = 0$ is called a \textbf{\textit{null set}}

\dfn If a statement about points $x\in X$ is true except for $x$ in some null set, we say that it is true \textbf{\textit{almost everywhere}} (abbreviated \textbf{\textit{a.e.}}), or for \textbf{\textit{almost every}} $x$.

\dfn A measure $\mu$ whose domain in $\MM$ includes all subsets of null sets is called \textbf{\textit{complete}}.

\vs 

\begin{thm}\label{1.9}
Suppose that $(X, \MM, \mu)$ is a measure space. Let $\NN = \{N\in \MM\ |\ \mu(N) = 0\}$ and $\ol\MM = \{E\cup F\ |\ E\in \MM\text{ and } F\subset N\text{ for some } N\in \NN\}$. Then $\ol\MM$ is a $\sig$-algebra and there is a unique extension $\ol\mu$ of $\mu$ to a complete measure on $\ol\MM$.
\end{thm}

\vs 

\dfn The measure in \autoref{1.9} is called the \textbf{\textit{completion}} of $\mu$, and $\ol\MM$ is called the \textbf{\textit{completion}} of $\MM$ with respect to $\mu$.

\dfn An \textit{\textbf{outer measure}} of an nonempty set $X$ is a function $\mu^*:\PP(X)\ra [0,\infty]$ that satisfies
\begin{itemize}
\item $\mu^*(\es) = 0$,
\item $\mu^*(A)\leq \mu^*(B)$ if $A\subset B$,
\item $\mu^*(\bigcup_1^\infty A_j) \leq \sum_1^\infty \mu^*(A_j)$.
\end{itemize}

\vs 

\begin{prop}
Let $\EE\subset \PP(X)$ and $\rho:\EE\ra [0,\infty]$ be such that $\es\in \EE, X\in \EE$, and $\rho(\es) = 0$. For any $A\subset X$, define
\[\mu^*(A) = \inf\left\{\sum_1^\infty \rho(E_j)\ :\ E_j\in \EE\ and\ A\subset\bigcup_1^\infty E_j\right\},\]
then $\mu^*$ is an outer measure.
\end{prop}

\vs 

\dfn Let $\mu^*$ be an outer measure on a set $X$. A set $A\subset X$ is called \textbf{\textit{$\boldsymbol{\mu^*}$-measurable}} if 
\[\mu^*(E) = \mu^*(E\cap A) + \mu^*(E\cap A^c)\text{ for all $E\subset X$}.\]

\vs 

\begin{thm}\label{thm: Carathedory}\textbf{(Carath\'edory's Theorem)}
if $\mu^*$ is an outer measure on $X$, the collection $\MM^*$ of $\mu^*$-measureable sets is a $\sig$-algebra, and the restriction of $\mu^*$ to $\MM$ is a complete measure.
\end{thm}

\vs 

\dfn If $\A\subset \PP(X)$ is an algebra, a function $\mu_0:\A\ra [0,\infty]$ is called a \textit{\textbf{premeasure}} if 
\begin{itemize}
\item $\mu_0(\es) = 0$,
\item if $\{A_j\}_1^\infty$ is a sequence of disjoint sets in $\A$ such that $\bigcup_1^\infty A_j\in \A$, then $\mu_0(\bigcup_1^\infty A_j) = \sum_1^\infty \mu_0(A_j)$.
\end{itemize}

\vs\vs

\setcounter{thm}{12}
\begin{prop}
Let $\mu_0$ be a premeasure on $\A$ and define
\[\mu^*(E) = \inf\left\{\sum_1^\infty \mu_0(A_j)\ :\ A_j\in \A\ and\ E\subset\bigcup_1^\infty A_j\right\}.\]
Then we have that
\begin{itemize}
\item $\mu^*|_\A = \mu_0$;
\item every set in $\A$ is $\mu^*$-measurable.
\end{itemize}
\end{prop}

\vs 

\begin{thm}
Let $\A\subset \PP(X)$ be an algebra, $\mu_0$ a premeasure on $\A$, and $\MM$ the $\sig$-algebra generated by $\A$. There exists a measure $\mu$ on $\MM$ whose restriction to $\A$ is $\mu_0$ --- namely, $\mu = \mu^*|_\MM$ where $\mu^*$ is the same as in the previous proposition. If $\nu$ is another measure on $\MM$ that extends $\mu_0$, then $\nu(E)\leq \mu(E)$ for all $E\in \MM$, with equality when $\mu(E) < \infty$. If $\mu_0$ is $\sig$-finite, then $\mu$ is the unique extension of $\mu_0$ to a measure on $\MM$.
\end{thm}

\vs 

\dfn Measures on $\R$ whose domain is the Borel $\sig$-algebra $\BB_\R$ are called \textbf{\textit{Borel measures}} on $\R$.

\textbf{Notation.} We shall call all subsets of $\R$ of the form $(a,b]$, $(a,\infty)$, or $\es$ where $-\infty\leq a < b < \infty$ \textit{\textbf{h-intervals}} (h for ``half-open'').

\vs 

\begin{prop}
Let $F:\R\ra \R$ be increasing an right continuous. If $(a_jb_j]$ (j = 1..n) are disjoint h-intervals, let 
\[\mu_0\lp \bigcup_1^n(a_jb_j]\rp = \sum_1^n[F(b_j) - F(a_j)],\]
and let $\mu_0(\es) = 0$. Then $\mu_0$ is a premeasure on the algebra $\A$ the collection of finite disjoint unions of h-intervals. 
\end{prop}

\vs 

\begin{thm}
If $F:\R\ra \R$ is any increasing, right continuous function, there is a unique Borel measure $\mu_F$ on $\R$ such that $\mu_F((a,b]) = F(b) - F(a)$ for all $a,b$. If $G$ is another such function, we have $\mu_F = \mu_G$ if and only if $F-G$ is constant. Conversely, if $\mu$ is a Borel measure on $\R$ that is finite on all bounded Borel sets and we define
\[F(x) = \begin{cases}
\mu((0,x])  & \text{if}\ x > 0,\\
0 & \text{if}\ x = 0,\\
-\mu((-x, 0]) & \text{if}\ x< 0,
\end{cases}\]
then $F$ is increasing and right continuous, and $\mu = \mu_F$.
\end{thm}

\vs 

\dfn Let $F$ be a increasing, right continuous function, and let $\mu_F$ be its associated Borel measure. Then the completion $\ol\mu_F$ (which will often be denoted by $\mu_F$ as well) is called the \textbf{\textit{Lebesgue-Stieltjes}} measure associated to $F$.

\textbf{Notation.} For the remainder of this section, we will fix a complete Lebesgue-Stieltjes measure $\mu$ on $\R$ associated to the increasing, right continuous function $F$, and we denote $\MM_\mu$ to be the domain of $\mu$.

\vs 

\begin{lem}
For any $E\in \MM_u$,
\[\mu(E) = \inf\left\{\sum_1^\infty\mu((a_j, b_j))\ :\ E\subset \bigcup_1^\infty(a_j, b_j)\right\}.\]
\end{lem}

\vs 

\begin{thm}
If $E\in \MM_\mu$, then
\begin{align*}
\mu(E                       ) &= \inf\{\mu(U)\ |\ U\supset E\text{ and $U$ is open}\}\\
&=\sup\{\mu(K)\ |\ K\subset E\text{ and $K$ is compact}\}.
\end{align*}
\end{thm}

\vs 

\begin{thm}
If $E\subset \R$, the following are equivalent:
\begin{enumerate}
\item[a.] $E\in \MM_u$
\item[b.] $E = V\backslash N_1$ where $V$ is a $G_\delta$ set and $\mu(N_1) = 0$
\item[c.] $E = H\cup N_2$ where $H$ is an $F_\sig$ set and $\mu(N_2) = 0$.
\end{enumerate}
\end{thm}

\vs 

\begin{prop}
If $E\in \MM_\mu$ and $\mu(E) < \infty$, then for every $\vep > 0$ there is a set $A$ that is a finite union of open intervals such that $\mu(E\vartriangle A) < \vep$.
\end{prop}

\vs 

\dfn Let $F:\R\ra \R$ be the identity function. Then the complete measure $\mu_F$ associated to $F$ is called the \textit{\textbf{Lebesgue measure}} which we will denote by $m$. The domain of $m$ is called the class of \textbf{\textit{Lebesgue measurable}} sets, and we shall denote it by $\LL$.


\vs

\begin{thm}
If $E\in \LL$, then $E + s\in \LL$ and $rE\in \LL$ for all $s,r\in \R$. Moreover, $m(E + s) = m(E)$ and $m(rE) = |r|m(E).$
\end{thm}



















