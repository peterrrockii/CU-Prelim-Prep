\newpage
\section{Integration}

\dfn If $(X, \MM)$ and $(Y,\NN)$ are measurable spaces, a mapping $f:X\ra Y$ is called $(\MM, \NN)$\textbf{-measurable}, or just \textbf{measurable} when $\MM$ and $\NN$ are understood, if $f\inv(E)\in \MM$ for all $E\in \NN$.

\vs

\begin{prop}
If $\NN$ is generated by $\EE$, then $f:X\ra Y$ is $(\MM, \NN)$-measurable if and only if $f\inv(E) \in \MM$ for all $E\in \EE$.
\end{prop}

\vs

\begin{cor}
If $X$ and $Y$ are metric (or topological) spaces, every continuous $f:X\ra Y$ is $(\BB_X, \BB_Y)$-measurable.
\end{cor}

\vs

\dfn If $(X,\MM)$ is a measurable space, or real -- or complex -- valued functions $f$ on $X$ will be called $\MM$\textbf{-measurable}, or just \textbf{measurable}, if it is $(\MM,\BB_\R)$ or $(\MM,\BB_\C)$ measurable. In particular, $f:\R\ra \C$ is \textbf{Lebesgue} (resp. \textbf{Borel}) \textbf{measurable} if it is $(\LL, \BB_\C)$ (resp. $(\BB_\R, \BB_\C)$) measurable; likewise for $f:\R\ra \R$.

\vs

\begin{prop}
If $(X,\MM)$ is a measurable space and $f:X\ra\RR$, the following are equivalent:
\begin{enumerate}[\hspace{1.4em}a.]
    \item $f$ is $\MM$-measurable.
    \item $f\inv((a, \infty))\in \MM$ for all $a\in \R$.
    \item $f\inv([a, \infty))\in \MM$ for all $a\in \R$.
    \item $f\inv((\infty, a))\in \MM$ for all $a\in \R$.
    \item $f\inv((\infty, a])\in \MM$ for all $a\in \R$.
\end{enumerate}
\end{prop}

\vs

\begin{prop}
Let $(X, \MM)$ and $(Y_\al, \NN_\al)_{\al\in A}$ be measurable spaces, $Y = \prod_{\al\in A}Y_\al,\ N = \bigotimes_{\al\in A} \NN_\al$, and $\pi_\al:Y\ra Y_\al$ the coordinate maps. Then $f:X\ra Y$ is $(\MM, \NN)$-measurable if and only if $f_al = \pi_al\circ f$ is $(\MM, \NN_\al)$-measurable for all $\al$.
\end{prop}

\vs

\begin{cor}
A function $f:X\ra \C$ is $\MM$-measurable if and only if $\Re(f)$ and $\Im(f)$ are $\MM$-measurable.
\end{cor}

\vs

\begin{prop}
If $f,g:X\ra\C$ are $\MM$-measurable, then so are $f + g$ and $fg$.
\end{prop}

\vs

\begin{prop}
If $\{f_j\}$ is a aequence of $\ol\R$-valued measurable functions on $(X, \MM)$, then the functions
\begin{alignat*}{2}
g_1(x) &= \sup_j f_j(x),\qquad g_3 &&= \limsup_{j\ra \infty} f_j(x)\\
g_1(x) &= \inf_j f_j(x),\qquad g_3 &&= \liminf_{j\ra \infty} f_j(x)
\end{alignat*}
are all measurable. If $f(x) = \lim_{j\ra \infty} f_j(x)$ exists for every $x\in X$, then $f$ is measurable.
\end{prop}

\vs

\begin{cor}
If $f,g:X\ra\ol\R$ are measurable, then so are $\max(f,g)$ and $\min(f,g)$.
\end{cor}

\vs

\begin{cor}
If $\{f_j\}$ is a sequence of complex-valued measurable functions and $f(x) = \lim_{j\ra \infty} f_j(x)$ exists for all $x$, then $f$ is measurable.
\end{cor}

\vs

\dfn If $f:X\ra \ol\R$, we define the \textbf{positive} and \textbf{negative parts} of $f$ to be
\[f^+(x) = \max(f(x), 0),\qquad f^-(x) = \max(-f9x), 0).\]
If $f:X\ra \C$, we define its \textbf{polar decomposition} by
\[f = (\text{sgn}(f) )|f|, \quad \text{where}\quad \text{sgn}(z) = \begin{cases} z/|z| & \text{if } z\neq 0, \\ 0 & \text{if } z = 0.\end{cases}\]

\vs

\dfn Suppose that $(X,\MM)$ is a measurable space. If $E\subset X$, the \textbf{characteristic function} $\chi_E$ of $E$ is defined by 
\[\chi_E(x) = \begin{cases} 1 &\text{if }x\in E,\\ 0 & \text{if }x\nin E.\end{cases}\]

\vs

\dfn A \textbf{simple function} of $X$ is a finite, linear combination, with complex coefficients, of characteristic functions of sets in $\MM$. Equivalently $f:X\ra \C$ is simple if and only if $f$ is measurable and the range of $f$ is a finite subset of $\C$. The \textbf{standard representation} of $f$ is given by
\[f = \sum_{j = 1}^n z_j \chi_{E_j}, \text{ where $E_j = f\inv(\{z_j\})$ and range$(f) = \{z_1,\ldots,z_n\}$.}\]

\vs

\begin{thm}
Let $(X, \MM)$ be a measurable space.
\begin{enumerate}[\hspace{1.4 em} a.]
    \item If $f:X\ra [0,\infty]$ is measurable, there is a sequence $\{\vphi_n\}$ of simple functions such that $0 \leq \vphi_1\leq \vphi_2\leq \cdots\leq f,\ \vphi \ra f$ pointwise, and $\vphi\ra f$ uniformly on any set on which $f$ is bounded.
    \item If $f:X\ra \C$ is measurable, there is a sequence $\{\vphi_n\}$ of simple functions such that $0 \leq |\vphi_1|\leq |\vphi_2|\leq \cdots\leq |f|,\ \vphi \ra f$ pointwise, and $\vphi\ra f$ uniformly on any set on which $f$ is bounded.
\end{enumerate}
\end{thm}

\vs

\begin{prop}
Let $(X, \MM, \mu)$ be a measure space. The following statements are true if and only if $\mu$ is a complete.
\begin{enumerate}
    \item If $f$ is measurable and $f(x) = g(x)$ $\mu$-a.e. then $g(x)$ is measurable.
    \item If $\{f_n\}$ is a sequence of measureable functions and $f$ is a unction on $X$ such that $\us{n\ra\infty}{\lim} f_n(x) = f(x)$ $\mu$-a.e., then $f$ is measurable.
\end{enumerate}
\end{prop}

\vs

\begin{prop}
If $(X, \MM, \mu)$ is a measure space and $\ol X, \ol \MM, \ol \mu)$ is its completion, then if $f$ is a $(\ol X, \ol \MM, \ol\mu)$ is a measureable function, there exists a function $g$ that is measureable in $(X, \MM, \mu)$ such that $f(x) = g(x)$ $\ol\mu-a.e.$.
\end{prop}

\vs
\textbf{Corollary.} If $f:R\ra [-\infty, \infty]$ is Lebesgue measurable in $(\R, \LL, m)$ then there is a Borel measurable function $g:R\ra [-\infty, \infty]$ such that $f(x) = g(x)$ $\ol\mu-a.e.$.


\dfn Let $(X, \MM, \mu)$ be a be a measure space. Let $\vphi:X\ra \C$. Recall that if $\vphi$ is simple if $\vphi$ is measureable and has finite range. In this case we can write
\[\vphi(x) = \sum_{i = 1}^n z_i\chi_{E_i}\]
where $\{z_i\}_1^n$ is the range of $\vphi$, and $\bigsqcup E_i = X$ with $E_i = \vphi\inv(\{z_i\})\in \MM$. Now suppose that $\vphi$ has non-negative range. 
\[\vphi(x) = \sum_{i = 1}^n a_i\chi_{E_i}\]
We define $\int_X \vphi d\mu$ (the integral of $\vphi$ with respect to $\mu$) by
\[\int_X \vphi = \int_X\sum_{i=1}^n a_i\chi_{E_i}d\mu = \sum_{i = 1}^n a_i\cdot \mu(E_i)\]
with the following arithmatic rules
\[a\cdot \infty = \begin{cases} 0 & a = 0 \\ \infty  & a > 0\end{cases} \qquad a + \infty = \infty\]
We have \hl{$L^+ = \{f:X\ra [0,\infty]\ :\ f \text{ is measureable}\}$}. We will define for $f\in L^+$. 
\[\int_X f\ d\mu = \sup\left\{\int_X\vphi\ : \ \vphi \text{ simple }, \vphi(x) \leq f(x)\right\}\]

\vs

\dfn For $\vphi$ simple $\vphi:X\ra [0,\infty)$ and $A\in \MM$ define
\[\int_A \vphi\ d\mu = \int_X\vphi\chi_A\ d\mu = \int_X\lp\sum_1^n a_i\chi_{E_i}\rp \chi_A\  d\mu = \int_X\sum_1^n a_i \chi_{E_i\cap A}\ d\mu = \sum_1^n a_i \mu(E_i\cap A).\]

\vs

\begin{prop}
Let $(X,\MM,\mu)$ be a measure space. Let $\vphi$ and $\psi$ be non-negative simple functions on $X$. Then 
\begin{enumerate}[\hh (a)]
    \item If $c\geq 0$, $\int_X c\vphi\ d\mu = c\int_x\vphi\ d\mu.$
    \item $\int_X(\vphi + \psi)\ d\mu = \int_X\vphi\ d\mu + \int_X \psi\ d\mu$.
    \item If $\psi(x) \leq \phi(x)$ for all $x\in X$ then 
    \[0\leq \int_X\psi(x)\ d\mu \leq \int_X \vphi(x)\ d\mu\]
    \item The map $A\mapsto \int_A\vphi\ d\mu$ defines a measure $\mu_\vphi$ on $(X, \MM)$.
\end{enumerate}
\end{prop}


\textbf{Proposition (Used in Hw).} Let $f,g:X\ra \ol\R$ are measurable. Then $f + g$ is measurable. 

\begin{proof} Since $f$ and $g$ are measurable, we know by \textcolor{red}{Theorem 2.10} that there are sequences of simple functions $\{\vphi_n\}$ and $\{\psi_n\}$ such that 
\[f = \lim_{n\ra\infty} \vphi_n\quad\text{and}\quad g = \lim_{n\ra\infty} \psi_n.\]
It then follows that for any $x\in X$, 
\[(f + g)(x) = \lim_{n\ra\infty} \vphi_n(x) + \psi_n(x)\]
and since the sum of simple functions remains simple, we get that $f + g$ is the (pointwise) limit of simple functions $\{\vphi_n + \psi_n\}$ all of which are measureable. So $f + g$ is the (pointwise) limit of measurable functions, and by \textcolor{red}{Proposition 2.7} $f + g$ is measurable.
\end{proof}

\vs

\dfn For $(X, \MM, \mu)$ a measure space and $f\in L^+$, recall we define 
\[\int_X f = \sup\left\{\int_X\vphi\ d\mu : \vphi\in L^+,\  \vphi \text{simple},\ \vphi(x) \leq f(x),\  \forall x\in X\right\}.\]
Let
\[N_f = \left\{ \int_X\vphi\ d\mu,\ \vphi \in L^+,\ \vphi\text{ simple},\ \vphi(x)\leq f(x),\ \forall x\in X\right\}\]
Then $\int_x f\ d\mu = \sup N_f\geq 0$. If $\psi\in L^+$ and $\psi$ is simple, then inf $\vphi$ is nonnegative and simple, and $\vphi(x) \leq \psi(x)$ for all $x\in X$. Then by \textcolor{red}{Proposition 2.13(c)} we have that 
\[\int_X\vphi\ d\mu \leq \int_x\psi\ d\mu.\]
So 
\[\int_X\psi\ d\mu = \sum_{j = 1}^n b_j \mu(F_j)\]
is a upper bound for $N_\psi$. So

\[\sup N_\psi \leq \sum_{j = 1}^n b_j\mu(F_j) = \int_X\phi\ d\mu.\]
\begin{thm}\textbf{(The Monotone Convergence Theorem)}
If $\{f_n\}$ is a sequence in $L^+$ such that $f_j \leq f_{j + 1}$ for all $j$, and $f = \lim_{n\ra \infty} f_n (=sup_n f_n)$, then 
\[\int f = \int\lim_{n\ra\infty} f_n = \lim_{n\ra \infty}\int f_n.\]
\end{thm}


\vs

\begin{thm}
Let $\{f_j\}_{j \in J}$ be a finite or countably infinite sequence of functions in $L^+$. Then 
\[\int_X\sum f_j\ d\mu = \sum \int_X f_j\ d\mu.\]
\end{thm} 

\vs

\begin{prop}
Let $(X, \MM, \mu)$ be a measure space, and let $f\in L^+$. Then
\[\int_X f\ d\mu = 0 \Leftrightarrow f(x) = 0\text{ a.e. on $X$.}\]
\end{prop}

\vs

\begin{cor}
Suppose that $\{f_n\}_{n = 1}^\infty\subset L^+$, $f\in L^+$, and that $f_n(x) \ra f(x)$ for almost every $x$. Then $\int f = \lim\int f_n$.
\end{cor}

\vs

\begin{thm}\textbf{(Fatou's Lemma)}
If $\{f_n\}$ is any sequence in $L^+$, then
\[\int_X(\liminf f_n) = \liminf\int_X f_n.\]
\end{thm}

\textit{Example:} $(\R, \LL, m)$ $f_n(x) = \frac{1}{n}$ for all $n\in \N$. Then $\lim f_n(x) = \lim_{n\ra \infty} \frac{1}{n} = 0$, so 
\[\int_\R \lim f_n\ dm = \int_\R = \ dm = 0 < \infty = \lim \int_\R \frac{1}{n}\ dm.\]


\dfn Let $(X, \MM, \mu)$ be a measure space.
\begin{enumerate}[\hspace{1em}(i)]
    \item Let $f\in L^+$. We way that $f$ is integrable if $\int_X f\ d\mu < \infty$.
    \item Let $F:X\ra [-\infty, \infty]$. We say that $f$ is integrable if $f^+$ and $f^-$ are both integrable (in the sense of (i)), and in this case, we define \[\int_X f\ d\mu = \int_x f^+\ d\mu - \int_x f^-\ d\mu.\]
    \item For $f:X\ra \C$ we asy that $f$ is integrable if $\Re(f)$ and $\Im(f)$ are both integrable, and in this case we define \[\int_X f\ d\mu = \int_X \Re(f)\ d\mu + i\int_X\Im(f)\ d\mu.\]
\end{enumerate}

\textbf{Note:} For (ii) and (iii) it is possible to show that $f$ is integrable if and only if $\int_X |f|\ d\mu < \infty$.

\hl{\textbf{Notation:}} The set of all integrable function is written $L^1,\ L^1(\mu),\ L^1(X, \mu)$.

\textbf{Exercise:} If $f\in L^1(\mu)$; $\al\in \C$, then $\al f\in L^1(\mu)$, and
\[\al\int_X f\ d\mu = \int_X\al f\ d\mu.\]

\textbf{Exercise:} If $f, g\in L^1(\mu)$, then 
\[\int_X (f + g)\ d\mu = \int_X f\ d\mu + \int_X g\ d\mu.\]

\vs

\begin{cor}
If $\{f_n\} \subset L^+,\ f\in L^+$, and $f_n\ra f$ a.e., then $\int f \leq \liminf \int f_n$.
\end{cor}

\vs

\begin{prop}
If $f\in L^+$ and $\int f < \infty$, then $\{x\ :\ f(x) = \infty\}$ is a null set and $\{x\ :\ f(x) > 0 \}$ is $\sig$-finite.
\end{prop}

\vs

\begin{prop}
The set of integrable real-valued functions on $X$ is a real vector space, and the integral is a linear functional on it.
\end{prop}

\vs

\begin{prop}
\hl{If $f\in L^1$, then $|\int f| \leq \int |f|$.}
\end{prop}

\vs

\begin{prop}\nl
\begin{enumerate}[\hspace{1em}(a)]
    \item Let $f\in L^1(\mu)$. Then $\{x\in X\ :\ f(x) \neq 0\}$ is $\sig$-finite.
    \item Let $f \in L^1(\mu)$. Then the following are equivalent:
    \begin{enumerate}[\hspace{1em}(i)]
        \item $\int_E f\ d\mu = \int_E g\ d\mu$ for all $E\in \MM$.
        \item $\int_X|f - g|\ d\mu = 0$.
        \item $f(x) = g(x)$ $\mu$-a.e.
    \end{enumerate}
\end{enumerate}
\end{prop}

\vs

\textbf{Proposition (Class).} For $f,g\in L^1(\mu)$, we write $f\sim g$ if $f(x) = g(x)$ $\mu$-a.e. By abuse of notation we write $L^1(\mu) = L^1(\mu)/\sim$ and $f = [f]$. Note by \textcolor{red}{Proposition 2.16} $\int_X |f|\ d\mu 0 0 \Leftrightarrow [f] = 0$. With this identification, we can make $L^1(\mu) = L^1(\mu)/\sim$ a metric space with
\[\rho(f,g) = \int_X|f - g|\ d\mu.\]

\vs
\begin{thm}\textbf{(The Dominated Convergence Theorem)}
Let $\{f_n\}$ be a sequence in $L^1$ such that $f_n\ra f$ a.e. and such that there is some $g\in L^1 \cap L^+$ such that $|f_n| \leq g$ a.e. for all $n$. Then $f\in L^1$ and $\int f = \lim_{n\ra \infty} \int f_n$.
\end{thm}

\vs

\begin{thm}
Suppose that $\{f_j\}$ is a sequence in $L^1$ such that $\sum_1^\infty \int |f_j| < \infty$. Then $\sum_1^\infty f_j$ converges a.e. to a function in $L^1$, and $\int\sum_1^\infty f_j = \sum_1^\infty \int f_j$.
\end{thm}

\vs

\begin{thm}
If $f\in l^1(\mu)$ and $\vep > 0$, there is an integrable simple function $\vphi = \sum a_j \chi_{E_j}$ such that $\int|f -\vphi|\ d\mu < \vep$. If $\mu$ is a Lebesgue-Stieltjes measure on $\R$, the sets $E_j$ in the definition of $\vphi$ can be taken to be finite unions of open intervals; moreover, there is a continuous function $g$ that vanishes outsided a bounded interval such that $\int |f - g|\ d\mu < \vep$.
\end{thm}

\vs

\begin{thm}
Let $(X, \MM, \mu)$ be a measure space. Suppose $f: X\times [a,b] \ra \C$ and suppose that for every $t\in a,b$, $f(\cdot, t)\in L^1(\mu)$, and $\int_X f(x, t)|_{d\mu(x)} < \infty$.
\begin{enumerate}
    \item Suppose there exists a $g_1\in L^1\cap L^+$ such $|f(x,t)| \leq g_1(x)$ for all $(x,t)\in X\times [a,b]$. Define $F(x) = \int_X f(x,t)\ d\mu$. If $\lim_{t\ra t_0} f(x, t) = f(x, t_0)$ for all $x\in X$, then $\lim_{t\ra t_0} F(t) - F(t_0)$. That is to say
    \[\lim_{t\ra t_0}\int_X f(x,t)\ d\mu = \int_X\lim{t\ra t_0} f(x,t)\ d\mu.\]
    
    \item Suppose $\frac{\partial f}{\partial t}$ exists for all $(x, t) \in X\times [a,b]$ and $|\frac{\partial f}{\partial x}| \leq g_2(x)$ for all $(x, t) \in X\times [a,b]$ where $g_2\in L^1\cap L^+$. Then $F(x) = \int_x f(x,t)\ d\mu(x)$ is differentiable on $[a,b]$ and $F\p(t_0) = \int_x \frac{\partial f}{\partial t}(x, t_0)\ d\mu(x)$ with
    \[\lim_{t\ra t_0} \frac{F(t) - F(t_0)}{t - t_0} = \lim_{t\ra t_0}\int_x \frac{f(x,t) = f(x,t_0)}{t-t_0}\ d\mu(x).\]
    \end{enumerate}
\end{thm}

\vs

\begin{thm}
Let $f:[a,b]\ra \R$ be a bounded function where $[a,b]$ is a closed and bounded interval
\begin{enumerate}
    \item If $f$ is Riemann integrable over $[a,b]$, then $f$ is Lebesgue measurable and Lebesgue integrable over $[a,b]$ with
    \[\int_a^b f(x)\ dx = \int_{[a,b]}f\ dm \]
    \item If $f$ is Riemann integrable over $[a,b]$ if and only if 
    \[m(\{x\in [a,b]\ :\ f \text{ is discontinuous at } x\}) = 0.\]
    
\end{enumerate}
\end{thm}

\vs

\textbf{Example:} (of the DCT) Consider 
\[\sum_{k = 0}^\infty x^{2k}\]
on $[0,1]$. Let 
\[f_n(x) = \sum_{k = 0}^n (-1)^k x^{2k} = \frac{1 + (-1)^{n + 1}x^{2(n + 1)}}{1 + x^2}.\]
Then $\lim f_n(x)$ exists for all $x\in [0,1)$ and diverges at $1$.
\[|f_n(x)| = \left|\frac{1 + (-1)^{n + 1}x^{2(n + 1)}}{1 + x^2}\right|\leq \frac{2}{1} = 2\]
on $[0,1]$. Note that
\[\lim_{n\ra \infty} f_n(x) = \lim_{n\ra \infty} \frac{1 + (-1)^{n + 1}x^{2(n + 1)}}{1 + x^2} = \frac{1}{1 + x^2}.\]
By the DCT, we then get that
\[\lim_{n\ra \infty} \int_{[0,1]}f_n(x) = \int_{[0,1]} \frac{1}{1 + x^2}\ dm = \frac{\pi}{4},\]
and
\[\lim_{n\ra \infty} \int_{[0,1]}f_n(x) = \lim_{n\ra \infty} \int_{[0,1]} \sum_{k = 0}^n (-1)^k x^{2k} = \sum_{k = 0}^\infty \frac{(-1)^k}{2k + 1}.\]
So 
\[\sum_{k = 0}^\infty \frac{(-1)^k}{2k + 1} = \frac{\pi}{4}.\]

\vs

\dfn Let $(X, \MM, \mu)$ be a measure space. Let $\{f_n: X\ra \C\}$ all be measurable and let $f:X\ra \C$ be measureable
\begin{enumerate}[\hspace{1em}(i)]
    \item We say that $f_n\ra f$ pointwise on $X$ if
    \[\lim_{n\ra \infty} f_n(x) = f(x)\]
    for all $x\in X$.
    \item $f_n\ra f$ uniformly if for all $\vep > 0$, there is some $N\in \N$ such that for all $n\geq N$, $|f_n(x) - f(x)| < \vep$ for all $x\in X$.
    \item We say $f_n\ra f$ in measure on $(X, \MM, \mu)$ if for all $\vep > 0$,
    \[\lim_{n\ra a} \mu(\{x\in X\ :\ |f_n(x) - f(x)| \geq \vep\}) = 0.\]
    \item Convergence in $L^1$, $\{f_n\} \in L^1$, $f\in L^1$, $[f_n]\ra [f]$ if 
    \[\lim_{n\ra\infty}\int_X |f_n - f|\ d\mu = 0.\]
\end{enumerate}

\vs

\textbf{Note:} The only implications that we can derive from these types of convergence are that (ii) $\Ra$ (i), and (iv) $\Ra$ (iii). \hl{THESE WILL BE ON THE EXAM!!!}

\vs

\begin{prop}
If $f_n \ra f$ in $L^1$, then $f_n\ra f$ in measure.
\end{prop}

\begin{proof}
Fix some $\vep > 0$. For each $n\in \N$, let $E_n = \{x\in X\ :\ |f_n(x) - f(x)| \geq \vep\}$. Well, we know that 
\[\int_X|f_n(x) - f(x)|\ d\mu \geq \int_X |f_n(X) - f(X)|\chi_{E_n}(x)\ d\mu = \int_X \vep \chi_{E_n}(x)\ d\mu.\]
This gives us that
\[\mu(E_n) \leq \frac{\int_X|f_n(x) - f(x)|\ d\mu}{\vep}\]
because there is some $N\in \N$ such that for all $n\geq \N$, $\int_X |f_n(x) - f(x)|\ d\mu < \vep$. So for $n\geq N$, $\mu(E_n) \leq \frac{\vep^2}{\vep}$ which establishes (iii).
\end{proof}

\vs

\dfn \hl{(Cauchy Sequence for convergence in measure)} Let $\{f_n\}_{n = 1}^\infty$ be a sequence of measureable functions on $(X, \MM, \mu)$. We say that $\{f_n\}_{n = 1}^\infty$ is a Cauchy sequence in the sense of convergence in measure if for all $\vep > 0$, there is an $N\in\N$ such that $m > n \geq N$
\[\mu(\{x\in X\ :\ |f_m(x) - f_n(x)|\geq \vep\}) < \vep.\]

\vs

\begin{thm}
Suppose that $\{f_n\}_{n = 1}^\infty$ are measureable on $(X, \MM, \mu)$ and that it is a Cauchy sequence with respect to convergence in measure. Then there is some measurable function $f$ such that $f_n \ra f$ in measure. Furthermore, there is a subsequence $\{f_{n_j}\}_{j = 1}^\infty \subset \{f_n\}_{n = 1}^\infty$ such that 
\[\lim_{j\ra \infty} f_{n_j}(x) = f(x)\]
pointwise a.e. on $(X, \MM, \mu)$. Moreover, if there exists some other measureable function $g$ such that $f_n \ra g$ in measure, we have that $f = g$ $\mu$-a.e.
\end{thm}

\vs

\textbf{\textit{There is an error}} in the theorem counter in the book.

\vs
\setcounter{thm}{31}

\begin{cor}
If $f_n \ra f$ in $L^1$, there is a subsequence $\{f_{n_j}\}$ such that $f_{n_j} \ra f$ a.e.
\end{cor}


\vs


\begin{thm}\textbf{(Egoroff)}
Suppose that $\mu(X) < \infty$, and $f_1, f_2, \ldots$ and $f$ are measureable, complex-valued functions on $X$ such that $f_n \ra f$ a.e. The for every $\vep > 0$ there exists $E \subset X$ such that $\mu(E) < \vep$ and $f_n \ra f$ uniformly on $E^c$.
\end{thm}

\vs

\dfn Let $(X, \MM, \mu)$ and $(Y, \NN, \nu)$ be measure spaces. We define a \textbf{measureable rectangle} to be a set of the form $A\times B$ where $A\in \MM$ and $B\in \NN$. 

\textbf{NB:} If we take all finite unions of the rectangles, we get an algebra, $\A$ and the $\sig$-algebra generated by this set is 
\[\A = \MM\otimes \NN\]

We get a premeasure on $\A$ if we define $\pi_0(E) := \sum_j \mu(A_j)\nu(B_j)$ for $E = \bigsqcup_j (A_j \times B_j)$. This $\pi$ behaves as a premeasure, and so it generates an outer measure $\pi^*$. This then restricts to a measure $\pi$ on $\MM\times \NN$ which we denote by $\mu\times \nu$. Moreover, if $(X, \MM, \mu)$ and $(Y, \NN, \nu)$ are $\sig$-finite, then $(X\times Y, \MM\otimes \NN, \mu \times \nu)$ is $\sig$-finite and is thus unique.

\vs

\dfn We define the $\boldsymbol{x}$\textbf{-section} $E_x$ and the $\boldsymbol{y}$\textbf{-section}, $E^y$ of a product space $X\times Y$ as follows:
\begin{align*}
     E_x &= \{y\in Y\ :\ (x,y) \in E\}\\
     E^y &= \{x\in X\ :\ (x,y) \in E\}.
\end{align*}
Also, if $f$ is a function on $X\times Y$, we can define the $x$-section $f_x$ and the $y$-section $f^y$ of $f$ by 
\[f_x(y) = f^y(x) = f(x, y).\]

\vs

\begin{prop}\nl
\begin{enumerate}
    \item If $E\in \MM\otimes \NN$, then $E_x\in \NN$ and $E^y\in \MM$ for all $x\in X$ and for all $y\in Y$.
    \item If $f$ is $\MM\otimes \NN$-measureable, then $f_x$ is $\NN$-measurable for all $x\in X$ and $f^y$ is $\MM$-measurable for all $y\in Y$.
\end{enumerate}
\end{prop}

\vs

\dfn Let $X$ be a set and let $\CC\subset \PP(X)$ be a collection of subsets of $X$. If $\CC$ is closed under increasing unions and decreasing intersections, then we call $\CC$ is called a \textbf{monotone class} of subsets of $X$.

\vs

\dfn Let $\EE \subset \PP(X)$, we have that the \textbf{monotone class generated by }$\boldsymbol\EE$ is the intersection of all monotone classes containing $\EE$.

\vs

\textbf{Note:} For collection of subsets $\EE$, the $\sig$-algebra generated by $\EE$ is a monotone class.

\vs

\begin{thm}\textbf{(The Monotone Class Lemma)}
If $\A$ is an algebra of subsets of $X$, then the monotone class $\CC$ generated by $A$ coincides with the $\sig$-algebra $\MM$ generated by $\A$.
\end{thm}

\vs

\begin{thm}
Suppose $(X, \MM, \mu)$ and $(Y, \NN, \nu)$ are $\sig$-finite measure spaces. If $E\in \MM\otimes \NN$, then the functions $x\mapsto \nu(E_x)$ and $y\mapsto \mu(E^y)$ are measureable on $X$ and $Y$, respectively, and
\[(\mu\times\nu)(E) = \int \nu(E_x)\ d\mu(x) = \int \mu(E^y)\ d\nu(y).\]
\end{thm}

\vs

\begin{thm}\textbf{(Fubini-Tonelli)}
Suppose that $(X, \MM, \mu)$ and $(Y, \NN, \nu)$ are $\sig$-finite measure spaces.
\begin{enumerate}
    \item (Tonelli) If $f\in L^+(X\times Y)$, then the functions $g(x) = \int f_x\ d\nu$ and $h(y) = \int f^y\  d\mu$ are in $L^+(X)$ and $L^+(Y)$, respectively, and
    \begin{align*}
    \int f\ d(\mu\times \nu) &= \int \left[\int f(x,y)\ d\nu(y) \right]\ d\mu(x)\\
    &= \int\left[ \int f(x,y)\ d\mu(x)\right] \ d\nu(y)
    \end{align*}
    
    \item (Fubini) If $f\in L^1(\mu \times \nu)$, then $f_x\in L^1(\nu)$, for a.e. $x\in X$, $f^y\in L^1(\mu)$ for a.e. $y\in Y$, and the a.e. defined functions $g(x) = \int f_x\ d\nu$ and $h(x) = \int f^y\ d\nu$are in $L^1(\mu)$ and $L^1(\nu)$, respectively, and the equation 
\end{enumerate}
\end{thm}

\vs

\setcounter{thm}{38}

\begin{thm}\textbf{(Fubini-Tonelli for Complete Measures)}
Suppose that $(X, \MM, \mu)$ and $(Y, \NN, \nu)$ are complete, $\sig$-finite measure spaces, and let $(X\times Y, \LL, \lambda)$ be the completion of $(X\times Y, \MM\otimes \NN, \mu\times \nu)$. If $f$ is $\LL$-measurable and either (a) $f\geq 0$ or (b) $ f\in \LL^1(\lambda)$, then $f_x$ is $\NN$-measurable for a.e. $x$ and $f^y$ is $\MM$-measurable for a.e. $y$, and in case (b) $f_x$ and $f^y$ are also integrable for a.e. $x$ and $y$. Moreover, $x\mapsto \int f_x\,d\nu$ and $y\mapsto \int f^y\,d\mu$ are measurable, and in case (b) also integrable, and
\[\int f\,d\lambda = \iint f(x,y)\,d\mu(x)\,d\nu(y) = \iint f(x,y)\,d\nu(y)\,d\mu(x).\]
\end{thm}









%end